\chapter{The Hill model}

In this chapter, Lie point symmetries for the Hill model will be calculated.
Since this work has already been done in \cite{ohlsson2020symmetry}, this will serve as a reproduction of their results.

%=============================================================================
\section{The Hill equation}

The Hill equation is an ODE that empirically describes cooperative binding to a multi-site protein.
It takes the form
\begin{equation} \label{eq:original-hill}
  \dv{Y}{t} = - v_\text{max} \frac{Y^n}{K_m + Y^n} = \Omega_n(t, Y), \quad
  n > 0.
\end{equation}
By nondimensionalizing both the substrate concentration with
\begin{equation}
  y = \frac{Y}{{K_m}^{1/n}}
\end{equation}
and the time with
\begin{equation}
  \tau = v_\text{max} \frac{t}{{K_m}^{1/n}},
\end{equation}
\cref{eq:original-hill} simplifies to
\begin{equation} \label{eq:hill}
  \dv{y}{\tau} = - \frac{y^n}{1 + y^n} = \omega_n(\tau, y), \quad
  n > 0,
\end{equation}
where \(y'\) denotes the derivative \(\dv{y}{\tau}\), and will do so for the remainder of this chapter.

%=============================================================================
\section{Finding a non-trivial infinitesimal generator}

The Hill equation clearly has a trivial Lie symmetry generator \(X=\partial_\tau\).
This can be seen by observing that \cref{eq:hill} has no \(\tau\) in either side's expression (remember that the equation is viewed in terms of its lift, residing in the jet space \(J^1\)).
However, one purpose of finding symmetries of differential equations is to distinguish between models.
Since the trivial symmetry \(X=\partial_\tau\) holds for all \(n\), it can not be used to compare different degrees of Hill models.
Finding a non-trivial Lie symmetry group is therefore of interest.

Any Lie point symmetry group must fulfill the linearized symmetry condition \ref{thm:linearized-first-order-symmetry}.
To solve \cref{eq:linearized-first-order-symmetry}, an Ansatz must be made.
The tangent fields of many Lie symmetries of ODE:s are linear in \(y\).
The Ansatz
\begin{equation}
  \pqty{\xi(\tau,y),\eta(\tau,y)} = \pqty{A(\tau) + B(\tau)y,C(\tau) + D(\tau)y}
\end{equation}
is therefore used.
Inserting the Ansatz into \cref{eq:linearized-first-order-symmetry}, and using the fact that \(y' = y^n / (1 + y^n)\), gives that
\begin{equation}
  (C' + D'y) + (A' + B'y - D) \frac{y^n}{1 + y^n} - B \frac{y^{2n}}{(1 + y^n)^2} =
  -n(C + Dy) \frac{y^{n-1}}{(1 + y^n)^2}.
\end{equation}
Multiplication with \((1 + y^n)^2\) in turn gives that
\begin{equation} \label{eq:hill-linear-symmetry}
  (C' + D'y)(1 + y^n)^2 + (A' + B'y - D)y^n(1 + y^n) - By^{2n} =
  -n(C + Dy) y^{n-1}.
\end{equation}
The equation can then be separated by powers of \(y\) into a system of equations.
However, since \(n\) is specified to be positive, care must be taken to ensure that the separation holds for all \(n>0\).

%-----------------------------------------------------------------------------
\subsection{Case of \texorpdfstring{\(n\neq1,2\)}{n not 1 or 2}}

When \(n>0\) and \(n\neq1,2\), all of the possible powers of \(y\) will be different.
\Cref{eq:hill-linear-symmetry} can the be separated into the system
\par\noindent % TODO: Put this into an environment
\begin{minipage}{\linewidth}
  \vspace{\abovedisplayskip}
  \begin{subequations}
    \begin{tabularx}{\textwidth}{l|LRN}
      \(y^0\)       & C'(\tau) &= 0                               & eq:det-any-0\\
      \(y^1\)       & D'(\tau) &= 0                               & eq:det-any-1\\
      \(y^{n-1}\)   & 0 &= -nC(\tau)                              & eq:det-any-nm1\\
      \(y^n\)       & 2C'(\tau) + A'(\tau) - D(\tau) &= -nD(\tau) & eq:det-any-n\\
      \(y^{n+1}\)   & 2D'(\tau) + B'(\tau) &= 0                   & eq:det-any-np1\\
      \(y^{2n}\)    & C'(\tau) A'(\tau) - D(\tau) - B(\tau) &= 0  & eq:det-any-2n\\
      \(y^{2n+1}\)  & D'(\tau) + B'(\tau) &= 0.                     & eq:det-any-2np1
    \end{tabularx}
  \end{subequations}
  \vspace{\belowdisplayskip}
\end{minipage}
From \cref{eq:det-any-nm1}
\begin{equation}
  C(\tau) = 0,
\end{equation}
which renders \cref{eq:det-any-0} irrelevant.
\Cref{eq:det-any-1} gives
\begin{equation}
  D(\tau) = c_1
\end{equation}
for a constant \(c_1\) (constants will further be denoted \(c_i\)).
\Cref{eq:det-any-n} can with the determined values for \(C\) and \(D\) be reduced to
\begin{equation}
  A'(\tau) = (1-n) c_1,
\end{equation}
which means that
\begin{equation}
  A(\tau) = c_2 + (1-n) c_1 x.
\end{equation}
Using \cref{eq:det-any-np1} and the determined value of \(D\),
\begin{equation}
  B(\tau) = c_3,
\end{equation}
which in turn renders \cref{eq:det-any-2np1} irrelevant.
Lastly, \cref{eq:det-any-2n} can be reduced to
\begin{equation}
  (1-n) c_1 - c_1 - c_3 = 0,
\end{equation}
which means that
\begin{equation}
  c_3 = -n c_1.
\end{equation}
All tangent fields given the Ansatz must therefore be on the form
\begin{equation}
  \pqty{\xi(\tau,y),\eta(\tau,y)} = 
  \pqty{\tilde{c}_1 + \tilde{c}_2 \left( (1-n) x - n y \right), \tilde{c}_2 y}.
\end{equation}

%-----------------------------------------------------------------------------
\subsection{Case of \texorpdfstring{\(n=1\)}{n is 1}}

When \(n=1\), \(y^0 = y^{n-1},\), \(y^1 = y^n,\) and \(y^{n+1} = y^{2n},\).
\Cref{eq:hill-linear-symmetry} therefore be separated into the system
\par\noindent % TODO: Put this into an environment
\begin{minipage}{\linewidth}
  \vspace{\abovedisplayskip}
  \begin{subequations}
    \begin{tabularx}{\textwidth}{l|LRN}
      \(y^0\) & 
      C'(\tau) &= -nC(\tau)                                             &
      eq:det-1-0\\

      \(y^1\) &
      D'(\tau) + 2C'(\tau) + A'(\tau) - D(\tau) &= -nD(\tau)            &
      eq:det-1-1\\

      \(y^2\) &
      2D'(\tau) + B'(\tau) + C'(\tau) + A'(\tau) - D(\tau) - B(\tau) &= 0 & 
      eq:det-1-np1\\

      \(y^3\) &
      D'(\tau) + B'(\tau) &= 0.                                         &
      eq:det-1-2np1
    \end{tabularx}
  \end{subequations}
  \vspace{\belowdisplayskip}
\end{minipage}
