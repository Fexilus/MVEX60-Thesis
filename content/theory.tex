\chapter{Theory}

In this chapter, the mathematical theory used in the thesis will be presented.

%=============================================================================
\section{Lie point symmetries}

% TODO: Write lie symmetry introduction
Given a \dots

\dots

\begin{equation} \label{eq:first-order-ode}
  \dv{y}{x} = \omega(x,y)
  \implies
  \dv{\hat{y}}{\hat{x}} = \omega(\hat{x},\hat{y})
\end{equation}

\dots

The symmetry condition can then be formulated as: % FIXME: not proper wording
\begin{defn}[Point symmetry] \label{defn:first-order-symmetry}
  A transformation \(\Gamma: (x,y) \mapsto (\hat{x}(x,y), \hat{y}(x,y))\) is a point symmetry of the ODE \ref{eq:first-order-ode} if
  \begin{equation}
    \dv{y}{x} = \omega(x,y)
    \implies
    \dv{\hat{y}}{\hat{x}} = \omega(\hat{x},\hat{y}).
  \end{equation}
\end{defn}
Using the total derivative and the definition of the ODE \ref{eq:first-order-ode},
\begin{equation}
  \dv{\hat{y}}{\hat{x}} = 
  \frac{\tD{x} \hat{y}(x,y)}{\tD{x} \hat{x}(x,y)} =
  \frac{\partial_x \hat{y} + y' \partial_y \hat{y}}{\partial_x \hat{x} + y' \partial_y \hat{x}} =
  \frac{\partial_x \hat{y} + \omega(x,y) \partial_y \hat{y}}{\partial_x \hat{x} + \omega(x,y) \partial_y \hat{x}},
\end{equation}
which means that \cref{defn:first-order-symmetry} can be written as
\begin{lem}
  A transformation \(\Gamma: (x,y) \mapsto (\hat{x}(x,y), \hat{y}(x,y))\) is a point symmetry of the ODE \ref{eq:first-order-ode} if
  \begin{equation}
    \frac{\partial_x \hat{y} + \omega(x,y) \partial_y \hat{y}}{\partial_x \hat{x} + \omega(x,y) \partial_y \hat{x}} = \omega(\hat{x},\hat{y})
  \end{equation}
\end{lem}

%-----------------------------------------------------------------------------
\subsection{Infinitesimal generators}

% TODO: Write about infinitesimal generators
To generalize the notation, a differential operator
\begin{equation}
  X = \xi(x,y) \partial_x + \eta(x,y) \partial_y,
\end{equation}
called an infinitesimal generator, can be introduced.