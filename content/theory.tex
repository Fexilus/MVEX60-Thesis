\chapter{Mathematical theory of symmetries}

In this chapter, the mathematical theory used in the thesis will be presented.
The theory mostly concerns symmetry groups in the setting of ordinary differential equations (ODE:s), and systems thereof.
To aid readers new to this subject and with varying familiarity of related mathematical subjects, the theory is presented in a straight forward fashion, avoiding unnecessary generalizations and proofs of most results.
The mathematical theory and notation used in this and subsequent chapters is based on the works of three authors.
For readers wishing to explore the subject further, the sources are presented here in short.

The first source \cite{hydon2000symmetry} by \citeauthor{hydon2000symmetry}, serves as a great introductory text on the subject of symmetry methods.
For readers unfamiliar with the fields of differential geometry and Lie algebras wishing to use methods similar to the ones in this thesis, the book can be helpful for getting of the ground.
This is mainly due to fact that theory is only introduced on a need to know basis throughout the book, and as such readers seeking the mathematical rigor lacking in this text will not find it there.

The second sources \cite{olver1993applications,olver1995equivalence} by \citeauthor{olver1993applications}, on the other hand treats the subject stringently and serves as good reference points for readers wishing to understand the theory in more depth.
\cite{olver1993applications} focuses more on the specific application to differential equations while \cite{olver1995equivalence} puts more focus on the geometric concepts involved.

The third source \cite{ovsiannikov1982group} by \citeauthor{ovsiannikov1982group}, deals with group classification, a slightly more advanced subject on which the technique developed in \cref{ch:param-ind} is based.
Due to its earlier publication date, the book treats some subjects in a less modern way than the works by \citeauthor{hydon2000symmetry} and \citeauthor{olver1993applications}, and is therefore not recommended for readers unfamiliar with the concerned theory.

The notation in this thesis will be based on the notation in \cite{hydon2000symmetry}, employing partial notational concepts from \cite{olver1995equivalence} and \cite{ovsiannikov1982group} when further clarity is desired.

%=============================================================================
\section{Point symmetries}

In mathematics, a symmetry is a transformation that preserves some structure.
The word symmetry is used since the mathematical concept of symmetry encompasses what is meant by symmetry in everyday speech.
A symmetric face is a face such that when mirrored it look the same.
Here, mirroring is the transformation and \enquote{looking the same} is the structure.
In mathematics, however, the structure must be well defined, or in more common language must be a statement that has a clear distinction between being true or false.
The transformation can also be chosen more freely, beyond the transformations usually implied when talking about symmetries in everyday speech.

The most common example of a mathematical symmetry is rotating and flipping a triangle.
The structure of the triangle can be stated as the location of all the vertices and the length of the edges between specific vertices.
Rotating an equilateral triangle 120 or 240 degrees around its center will change the location of each individual vertex, but the positions of the set of all vertices will remain the same.
Since the edges are unchanged relative to the points, these rotations thus constitute a symmetry of the triangle.
It is worth noting that the rotational transformations map points from the 2-dimensional plane to the two dimensional plane or, stated in mathematical terms, the rotations are mappings \(\Gamma: \reals^2 \to \reals^2\).

In the methods used in this thesis, transformations similar to such a \(\Gamma\) are considered, but the structure of a triangle is instead replaced with the structure of a differential equation.
Initially, assume the differential equation is an ODE of the first order and can thus be written on the form
\begin{equation} \label[ode]{eq:first-order-ode}
  \dv{y}{x} = \omega(x,y).
\end{equation}
The structure of a first order ODE is thus defined by the function \(\omega\).
\(\omega\) is hence related to an infinite set of solutions \(u(x)\) that all fulfill \cref{eq:first-order-ode}.
Geometrically, a solution (or any scalar function of \(x\) for that matter) can be thought of as a set \(\gamma_u\) of points in the \(xy\)-plane, which constitutes a curve, for which \(y = u(x)\) holds for all \(x\).
Not every curve in the \(xy\)-plane corresponds to a function however.
A curve corresponds to a function if and only if it is transverse, that is: the tangent of the curve never points in only the \(y\)-direction.
Additionally for a given ODE, every point in the \(xy\)-plane belongs to exactly one such curve \(\gamma_u\): the curve of the solution that runs through that point.

A symmetry of a first order ODE \labelcref{eq:first-order-ode} is a transformation that preserves the solution curves.
This means that if two points \(\pqty{x_1,y_1}\) and \(\pqty{x_2,y_2}\) belong to the same solution curve \(\gamma\), the transformed points \(\Gamma(x_1,y_1)\) and \(\Gamma(x_2,y_2)\) must belong to the same solution curve, denoted by \(\Gamma\gamma\), for the transformation \(\Gamma\) to be a symmetry.
This property can be defined as:
\begin{defn}[Point symmetry] \label{defn:first-order-symmetry}
  A transformation 
  \begin{equation}
    \Gamma: \pqty{x,y} \mapsto \pqty{\hat{x}(x,y), \hat{y}(x,y)}
  \end{equation}
  is a point symmetry of \cref{eq:first-order-ode} if
  \begin{equation}
    \dv{y}{x} = \omega(x,y)
    \implies
    \dv{\hat{y}}{\hat{x}} = \omega(\hat{x},\hat{y}).
  \end{equation}
\end{defn}
The \enquote{point} in point symmetry refers to the fact that the transformation \(\Gamma\) only acts on the points belonging to the solution curves and nothing else.
This must not always be the case when generalizing the theory, but those generalizations will not be touched upon in this thesis.

%-----------------------------------------------------------------------------
\subsection{Jet spaces} \label{subsec:jet-spaces}

The observant reader might have already noted that there is some need of clarification for \cref{defn:first-order-symmetry}.
While the transformation \(\Gamma\) treats \(y\) as a point in \(\reals\), \cref{eq:first-order-ode} treats \(y\) as a differentiable function of \(x\), or in other words an element \(y(x)\) of the set of once continuously differentiable functions \(\mathcal{C}^1(\reals)\).
In many contexts this abuse of notation could be accepted without further remarks, but when using symmetry methods the subtleties of this operation is key to the calculations performed.
To avoid confusion while introducing the subject in this subsection, the point representation will be denoted by \(y\) while the function representation will be denoted by \(f(x)\).

The \(xy\)-plane previously mentioned can be seen as consisting of two components: the independent space \(X \simeq \reals\) and the dependent space \(Y \simeq \reals\).
The plane then is the product space \(E = X \times Y \simeq \reals^2\).
It is in this plane \(E\), commonly called the total space, that the curves \(\gamma\) live.
To be able to formulate differential statements, the space \(Y_1 \simeq \reals\) is used.
The elements of \(Y_1\) will be denoted \(y'\) or \(y^{(1)}\), implying that the elements correspond to the first derivative \(f'(x)\) of the function \(f(x)\).
It is however important to note that this space is no more the space of derivative functions \(f'(x)\) than \(Y\) is the space of once continuously differentiable functions \(f(x)\); the implied relation between the two is merely aesthetic so far.
\(Y_1\) can be combined with \(E\) to create \(J_1 = J_1 E = E \times Y_1 = X \times \prolong{Y}{1} \simeq \reals^3\), the first order jet space of \(E\).

All differentiable functions \(f: X \to Y\) have a single equivalent function \(\prolong{f}{1}: X \to \prolong{Y}{1}\) called the first prolongation of \(f(x)\).
The prolongation is defined as
% FIXME: Avoid relax
\begin{equation}
  \prolong{f}{1}(x) = \qty(f(x), \dv{f}{x}\relax(x)).
\end{equation}
In extension to this concept, a smooth transverse curve \(\gamma \subset E\) has a prolongation which is a curve \(\prolong{\gamma}{1} \subset J_1\) defined by
\begin{equation}
  \prolong{\gamma}{1} = \qty{\qty(x, \prolong{f}{1}(x)): x \in X},
\end{equation}
where \(f\) is the function corresponding to the curve \(\gamma\).

Using these tools, \cref{eq:first-order-ode} can be reformulated using jet spaces.
The purpose of the ODE is to define some set of solutions, and as such the reformulation should define that same set of solutions.
Using jet-notation, the problem can be formulated as finding curves \(\prolong{\gamma}{1}\) in \(J_1\) satisfying
\begin{equation} \label{eq:first-order-jet-ode}
  \Delta(\prolong{z}{1}) = y' - \omega(x,y) = 0,
\end{equation}
as well as the condition that the curve is a prolongation of some curve \(\gamma\) corresponding to a continuously differentiable function.
The curves \(\gamma\) will then correspond to the solutions \(u(x)\) of \cref{eq:first-order-ode}.

While only first order ODE:s are handled in this thesis, some basic knowledge of symmetry methods for higher order ODE:s is required to be able to understand how the process of finding symmetries differ between first and higher order ODE:s.
An ODE of degree \(k\) can be written on the form
\begin{equation}
  \dv[k]{y}{x} = \omega\qty(x, y, \dv{y}{x}, \dots, \dv[k-1]{y}{x}),
\end{equation}
while higher order jet spaces \(J_k = J_k E\) take the form
\begin{equation}
  J_k = X \times Y \times Y_1 \times \dots \times Y_k
\end{equation}
with elements
\begin{equation}
  \prolong{z}{k} = \qty(x, y, y', \dots, y^{(k)}).
\end{equation}
An ODE of degree \(k\) can thus be formulated as finding curves in \(J_k\) that satisfy
\begin{equation}
  \Delta(\prolong{z}{k}) = y^{(k)} - \omega(x, y, y', \dots, y^{(k-1)}) = 0
\end{equation}
as well as the condition that the curves should be \(k\):th order prolongations
\begin{equation}
  \prolong{\gamma_u}{k} = \qty{\prolong{z}{k} = \qty(\fixed{x}, y, y', \dots, y^{(k)})\qq*{:} y^{(l)} = \dv[l]{u}{x}{(\fixed{x})}\qc \fixed{x} \in X\qc l = 0, \dots, k}
\end{equation}
of curves \(\gamma\) corresponding to functions \(u(x) \in \cont{k}{X}\).
Here \(y^{(0)} = y\) and \(\dv[0]{u}{x} = u(x)\).

%-----------------------------------------------------------------------------
\subsection{The total derivative}

Since the jet space \(J_k\) treats the variables \(y\), \(y'\), \dots, \(y^{(k)}\) as independent coordinates, differentiation in \(x\), denoted as \(\partial_x\), will not affect those terms.
While this is the desired behavior of a jet space, there is also a need to be able to mirror the behavior of differentiation of the functions \(y(x)\), \(y'(x)\), \dots, \(y^{(k)}(x)\) in \(x\).
Thus a differential operator
\begin{equation}
  \tD{x} = \partial_x + y' \cdot \partial_y + y'' \cdot \partial_{y'} + \dots + y^{(l)} \cdot \partial_{y^{(l-1)}} + \dots
\end{equation}
on \(J_k\) is introduced, called the total derivative.
\(\tD{x}\) will act on expressions in \(J_k\) in the same way that \(\dv{x}\) would act on corresponding expressions consisting of \(x\), \(y(x)\) and derivatives of \(y(x)\) in \(x\).

A change of variables
\begin{align}
  \hat{x} &= \hat{x}(x, y(x))\\
  \hat{y} &= \hat{y}(x, y(x)) = \hat{y}(\hat{x})
\end{align}
in the function space can be represented in the jet space using the total derivative.
Due to the chain rule,
\begin{equation}
  \dv{\hat{y}}{\hat{x}} = 
  \frac{\pdv{\hat{y}}{x} + \pdv{\hat{y}}{y} \dv{y}{x}}{\pdv{\hat{x}}{x} + \pdv{\hat{x}}{y} \dv{y}{x}}
\end{equation}
in the function space.
In order to keep the correspondence between the function view and the jet view, viewing the change of variables as a transformation
\begin{equation}
  \Gamma: \pqty{x,y} \mapsto \pqty{\hat{x}(x,y), \hat{y}(x,y)},
\end{equation}
the first prolongation of that transformation defined by
\begin{equation}
  \prolong{\Gamma}{1}: \pqty{x,y,y'} \mapsto \pqty{\hat{x}(x,y), \hat{y}(x,y), \frac{\tD{x} \hat{y}(x,y)}{\tD{x} \hat{x}(x,y)}}
\end{equation}
will retain the correspondence to the ODE in the changed variables.

Using the prolongation of the transformation, \cref{defn:first-order-symmetry} of a first order symmetry for \cref{eq:first-order-ode} can be restated in the first order jet space \(J_1\) for the corresponding \cref{eq:first-order-jet-ode}.
\begin{lem} \label{lem:simple-first-order-symmetry}
  A transformation \(\Gamma: \pqty{x,y} \mapsto \pqty{\hat{x}(x,y), \hat{y}(x,y)}\) is a point symmetry of \cref{eq:first-order-ode} if and only if
  \begin{equation} \label{eq:simple-first-order-symmetry}
    \frac{\partial_x \hat{y} + \omega(x,y) \partial_y \hat{y}}{\partial_x \hat{x} + \omega(x,y) \partial_y \hat{x}} = \omega(\hat{x},\hat{y})
  \end{equation}
  holds.
\end{lem} % FIXME: Requirement of diffeomorphism
Given a transformation \(\Gamma\), it is therefore easy to check whether it is a point symmetry of \cref{eq:first-order-ode}.
The reverse process (finding point symmetries of \cref{eq:first-order-ode}) is on the other hand not easy, as \cref{eq:simple-first-order-symmetry} will in all but the most trivial cases result in a non-linear PDE in with two unknown functions.
However, by restricting attention to a specific type of point symmetries, these conditions can be simplified.

%=============================================================================
\section{Lie point symmetries}

A common restriction when studying symmetries is to limit the scope of sought symmetries to Lie groups of symmetries.
A Lie group is a mathematical group\footnote{}, where the members of the group can be parametrized by one or several continuous parameters.% TODO: Write footnote on groups
A Lie group of transformations on \(E\) is thus a group with composition as the operator, where the transformations are parameterized by one or several continuous parameters.
If the transformations can be indexed by a single real parameter \(\epsilon \in \reals\), the Lie group is said to be a one-parameter (real) Lie group of transformations.
The transformations in such a Lie group can be parametrized as
\begin{equation} \label{eq:1p-lie-point-symmetry}
  \Gamma_\epsilon: \pqty{x,y} \mapsto \pqty{\hat{x}_\epsilon(x,y), \hat{y}_\epsilon(x,y)},
\end{equation}
where both \(\hat{x}_\epsilon(x,y)\) and \(\hat{y}_\epsilon(x,y)\) are smooth (and therefore differentiable) in \(\epsilon\) when fixing \(x\) and \(y\).
Additionally, the transformation parametrized by \(\epsilon = 0\) will be the identity transformation \(\Gamma_0: \pqty{x,y} \mapsto \pqty{x,y}\).

If all transformations in such a group are symmetries of a differential equation, the group is said to be a Lie point symmetry of the differential equation.
The following is a simple example of a one parameter Lie point symmetry.
\begin{exmp}
  The ODE
  \begin{equation} \label{eq:ex-a:ode}
    \dv{y}{x} = y = \omega(x, y)
  \end{equation}
  has several symmetries.
  One group of symmetries is
  \begin{equation} \label{eq:ex-a:transformation}
    \Gamma_\epsilon(x,y) =
    \pqty{\hat{x}_\epsilon(x,y), \hat{y}_\epsilon(x,y)} =
    \pqty{x + \epsilon, y}, \quad
    \forall \epsilon \in \reals
  \end{equation}
  This can be shown by viewing \cref{eq:ex-a:ode} in the jet space \(J_1\), resulting in
  \begin{equation}
    \frac{\partial_x \hat{y}_\epsilon + \omega(x,y) \partial_y \hat{y}_\epsilon}{\partial_x \hat{x}_\epsilon + \omega(x,y) \partial_y \hat{x}_\epsilon} =
    \frac{y}{1} =
    \hat{y}_\epsilon =
    \omega(\hat{x}_\epsilon,\hat{y}_\epsilon), \quad
    \forall \epsilon.
  \end{equation}
  So by \cref{lem:simple-first-order-symmetry} the transformations are point symmetries.
  By fixing \(\pqty{x,y}=\pqty{\fixed{x},\fixed{y}}\),
  \begin{equation}
    \dv{\epsilon} \pqty{\hat{x}_\epsilon(\fixed{x},\fixed{y}), \hat{y}_\epsilon(\fixed{x},\fixed{y})} =
    \pqty{\dv{\fixed{\hat{x}}}{\epsilon}, \dv{\fixed{\hat{y}}}{\epsilon}} =
    \pqty{1,0}
  \end{equation}
  for any \(\pqty{\fixed{x},\fixed{y}}\).
  This means that the transformations are smooth in \(\epsilon\), and thus constitute a one parameter Lie point symmetry.
\end{exmp}
While the Lie groups of point symmetries can be parametrized explicitly such as in \cref{eq:ex-a:transformation}, it is more useful to characterize the Lie group by its associated vector field
\begin{equation} \label{eq:tangent-field}
  \pqty{\xi(x,y), \eta(x,y)},
\end{equation}
also known as the tangent field of the transformation group.
The tangent field can be calculated pointwise by 
\begin{equation}
  \eval{\dv{\epsilon} \pqty{\hat{x}_\epsilon(\fixed{x},\fixed{y}), \hat{y}_\epsilon(\fixed{x},\fixed{y})}}_{\epsilon=0} = \pqty{\xi(\fixed{x},\fixed{y}), \eta(\fixed{x},\fixed{y})}
\end{equation}
for every fixed point \(\pqty{\fixed{x},\fixed{y}} \in E\).
The tangent field can be thought of as the vector field that points in \(E\) flow along when the parameter \(\epsilon\) of transformation \(\Gamma_\epsilon\) is increased.
% TODO: Write about relation to parameterized transformation

%-----------------------------------------------------------------------------
\subsection{The linearized symmetry condition}

Using the tangent field of a one-parameter Lie symmetry group, the symmetry condition as formulated in \cref{lem:simple-first-order-symmetry} can be further simplified.
\begin{lem} \label{lem:linearized-first-order-symmetry}
  A one-parameter Lie group of point transformations constitute a Lie point symmetry of \cref{eq:first-order-ode} if and only if
  \begin{equation}
    \partial_x \eta + (\partial_y \eta - \partial_x \xi) \omega - \partial_y (\xi) \omega^2 -
    \xi \partial_x \omega - \eta \partial_y \omega = 0,
  \end{equation}
  where \(\pqty{\xi(x,y), \eta(x,y)}\) is the tangent field of the Lie group.
\end{lem}
\begin{proof}
  The transformations \(\Gamma_\epsilon\) are point symmetries of \cref{eq:first-order-ode} if and only if
  \begin{equation} \label{eq:parametrized-symmetry-cond}
    \frac{\partial_x \hat{y}(x,y;\epsilon) + \omega(x,y) \partial_y \hat{y}(x,y;\epsilon)}{\partial_x \hat{x}(x,y;\epsilon) + \omega(x,y) \partial_y \hat{x}(x,y;\epsilon)} = \omega(\hat{x}(x,y;\epsilon),\hat{y}(x,y;\epsilon)).
  \end{equation}
  Deriving by \(\epsilon\),
  \begin{equation}
    \begin{split} \label{eq:derive-symmetry-cond}
      &\frac{\partial_x \dv{\hat{y}}{\epsilon} + \omega \partial_y \dv{\hat{y}}{\epsilon}}{\partial_x \hat{x} + \omega(x,y) \partial_y \hat{x}} - \frac{\left(\partial_x \dv{\hat{x}}{\epsilon}  + \omega \partial_y \dv{\hat{x}}{\epsilon}\right)\left(\partial_x \hat{y} + \omega(x,y) \partial_y \hat{y}\right)}{\left(\partial_x \hat{x} + \omega(x,y) \partial_y \hat{x}\right)^2} -\\
      &\quad- \dv{\hat{y}}{\epsilon} \partial_x \omega - \dv{\hat{y}}{\epsilon} \partial_y \omega = 0.
    \end{split}
  \end{equation}
  It is sufficient to show that this expression holds at \(\epsilon = 0\) for all \(x\) and \(y\), since evaluation in any \(\tilde{\epsilon} \neq 0\) will be equivalent to the expression for \(\epsilon = 0\) in the point \(\Gamma_{\tilde{\epsilon}}(x, y)\).
  Evaluation at \(\epsilon = 0\) yields \(\dv*{\hat{x}}{\epsilon} = \xi\) and \(\dv*{\hat{y}}{\epsilon} = \eta\).
  Additionally, \(\eval{\hat{x}(x,y;\epsilon)}_{\epsilon = 0} = x\) and \(\eval{\hat{y}(x,y;\epsilon)}_{\epsilon = 0} = y\) since the transformation parametrized by \(\epsilon = 0\) is always the identity transformation in a one-parameter Lie groups of transformations.
  \Cref{eq:derive-symmetry-cond} evaluates to
  \begin{equation} \label{eq:linearized-first-order-symmetry-result}
    \partial_x \eta + (\partial_y \eta - \partial_x \xi) \omega - \partial_y (\xi) \omega^2 
    -\xi \partial_x \omega - \eta \partial_y \omega = 0,
  \end{equation}
  and hence the proof is complete.
\end{proof}

%-----------------------------------------------------------------------------
\subsection{Invariant solutions and trivial symmetries}

An important property of a Lie group of symmetries is which, if any, solution curves \(\gamma\) are invariant under all transformations \(\Gamma_\epsilon\).
The curve being invariant means that the transformed solution curve \(\Gamma_\epsilon\gamma = \gamma\).
It should be noted that this does not mean that every point in \(\gamma\) need be transformed to itself; the equivalency must merely hold for the entire set \(\gamma\).

The invariance of a solution can be studied using the characteristic
\begin{equation}
  Q(x, y, y') = \eta(x, y) - y' \xi(x, y)
\end{equation}
of a Lie group of symmetries with tangent field \(\pqty{\xi(x,y), \eta(x,y)}\).
The characteristic can be seen the magnitude of the cross product between the tangent field and the direction of solution curves
\begin{equation}
  \tD{x} \pqty{x, y} = \pqty{1, y'}.
\end{equation}
As is known from linear algebra, the cross product of two vectors has magnitude 0 if and only if they are parallel.
So if the characteristic is equal to zero in a point \(\fixed{x}, \fixed{y}\), the tangent field is parallel with the tangent of the solution curve in that point.
For first order ODE:s, \cref{eq:first-order-jet-ode} always holds on solution curves.
Hence, the reduced characteristic
\begin{equation}
  \bar{Q}(x, y) = \eta(x, y) - \omega(x, y) \xi(x, y)
\end{equation}
can be used to find invariant solutions of a first order ODE under a specific Lie point symmetry.
For any solution curve \(\gamma_u \subset E\) corresponding to a solution \(u(x)\), the curve is invariant under a Lie point symmetry if and only if
\begin{equation}
  \bar{Q}(x, u(x)) = 0.
\end{equation}

A Lie group of symmetries for which all solution curves \(\gamma\) are invariant is called a trivial symmetry group.
For such a symmetry group, the reduced characteristic will be zero everywhere, or in other words
\begin{equation} \label{eq:trivial-symmetry-first-order}
  \bar{Q}(x, y) \equiv 0.
\end{equation}
Trivial symmetries are called trivial, since finding one is easy.
Simply let the tangent field \(\pqty{\xi(x,y), \eta(x,y)} = \pqty{1, \omega(x,y)}\).
The Lie transformation group is a point symmetry group, since \cref{lem:linearized-first-order-symmetry} is fulfilled.
And it is a trivial symmetry group since \cref{eq:trivial-symmetry-first-order} is always fulfilled.

%=============================================================================
\section{Generalizations to higher orders}

The theory so far discussed has dealt with single ODE:s of the first order.
Most systems of interest are however not modeled in this way.
Therefore, the theory need be extended in three ways: from ordinary differential equations to partial differential equations, from first order differential equations to \(k:th\) order differential equations and from single differential equations to systems of differential equations.
While this might seem like a large amount of generalization to do in one step, the notation needed to deal with the added complexity is common enough for all three generalizations that they are best dealt with together.

%-----------------------------------------------------------------------------
\subsection{The infinitesimal generator and prolongations}

To simplify the notation a differential operator
\begin{equation}
  X = \xi(x,y) \partial_x + \eta(x,y) \partial_y
\end{equation}
called an infinitesimal generator, can be introduced.
The infinitesimal generator acts on mappings from \(E\) and, since it is defined only by \(\xi\) and \(\eta\), serves as a characterization of a Lie group of transformations.
Applying the operator to a mapping \(f: E \to E\), one gets a measure of the change in \(f\) when continuously transforming the plane which it acts on, since
\begin{equation}
  \eval{\dv{\epsilon} f \circ \Gamma_\epsilon}_{\epsilon = 0} = X f,
\end{equation}
where \(\Gamma_\epsilon\) is a Lie group of symmetries with corresponding tangent field \(\pqty{\xi(x,y), \eta(x,y)}\).

Without much work, the currently established mechanics of Lie point symmetries can be restated compactly using the infinitesimal generator.
The only additional theory needed is a generalization of prolongations.
In \cref{subsec:jet-spaces} the concept of a prolonged function was introduced.
This allowed a function to be represented by a corresponding curve in a jet space \(J_k\).
Both mappings \(f: E \to E\) and operators on such mappings can also be prolonged, resulting in mappings \(\prolong{f}{k}: J_k \to J_k\) and operators on such mappings.

The prolongations of mappings \(f: E \to E\) should fulfill the criteria
\begin{equation}
  \prolong{f}{k}(\prolong{\gamma_u}{k}) = \prolong{f(\gamma_u)}{k},
\end{equation}
and thus 
\begin{equation}
  f^i(x, u) = \dv[i]{f(x, u(x))}{x}\qc i = 0, \dots, k,
\end{equation}
where \(f^i\) is the \(i\):th component of the prolonged mapping \(\prolong{f}{k}\), the 0:th component referring to the \(y\)-component of the original mapping \(f\).
Using the total derivative, this can be rewritten as
\begin{equation}
  f^i = \tD{x} f^{i-1} \qc i = 0, \dots, k.
\end{equation}
In an analogous way, the prolongations of operators on mappings \(f: E \to E\) should be consistent independent of the order the prolongations are realized. % TODO: Write this more concretely

The infinitesimal generator can be prolonged, still entirely defined by \(\xi\) and \(\eta\), to act on any given jet space \(J_k\).
For first order ODE:s the jet space is \(J_1\), and thus the first prolongation is needed, which takes the form
\begin{equation}
  \prolong{X}{1} =
  \xi(x,y) \partial_x + \eta(x,y) \partial_y + \prolongpart{\eta}{1}(x,y) \partial_{y'},
\end{equation}
where
\begin{equation}
  \prolongpart{\eta}{1}(x,y) =
  \eta_x(x,y) + (\eta_y(x,y) - \xi_x(x,y)) y' - \xi_y(x,y) \left(y'\right)^2,
\end{equation}
where the subscripts \(x\) and \(y\) denote the partial derivatives \(\partial_x\) and \(\partial_y\).
The linearized symmetry condition in \cref{lem:linearized-first-order-symmetry} can be rewritten using the infinitesimal operator.
\begin{lem} \label{lem:linearized-first-order-symmetry-infinitesimal}
  A Lie group of point transformations is a symmetry of \cref{eq:first-order-ode} if
  \begin{equation} \label{eq:linearized-first-order-symmetry}
    \eval{\prolong{X}{1}\left(y' - \omega(x,y)\right)}_{y' = \omega(x,y)} = 0,
  \end{equation}
  where \(\prolong{X}{1}\) is the first prolongation of the infinitesimal generator of the Lie group of point transformations.
\end{lem}
On this form, the linearized symmetry condition is easier to interpret:
\(\varepsilon(\prolong{z}{1}) = y' - \omega(x,y) = 0\) holds for a solution of \cref{eq:first-order-ode}.
Since the transformations of solution curves must be solution curves in order for the transformations to be a symmetry group, the value of \(\varepsilon(\prolong{z}{1})\) must not change.
By evaluating the expression at \(y' = \omega(x,y)\), it is ensured that this holds for any solution to \cref{eq:first-order-ode}.

By finding further prolongations of \(X\), the linearized symmetry condition can be extended to ODE:s of higher degrees.
The \(k\):th order prolongation of the infinitesimal generator is
\begin{equation}
  \prolong{X}{k} = \xi(x,y) \partial_x + \eta(x,y) \partial_y + \prolongpart{\eta}{1}(x,y) \partial_{y'} + \dots + \prolongpart{\eta}{k}(x,y) \partial_{y^{(k)}}
\end{equation}
where \(\prolongpart{\eta}{k}\) is defined recursively by
\begin{align}
  \prolongpart{\eta}{0} &= \eta\\
  \prolongpart{\eta}{k}(x,y) &= \tD{x}(\prolongpart{\eta}{k-1}) - y^{(k)} \tD{x} \xi.
\end{align}
\Cref{lem:linearized-first-order-symmetry-infinitesimal} will hold for higher order ODE:s, given that \(y'\) is replaced by \(y^{(k)}\) and \(\omega(x,y)\) with \(\omega(\prolong{z}{k-1})\), and that the infinitesimal generator is prolonged to the \(k\):th degree.

\section{Finding Lie point symmetries}

% TODO: Write about basis for generators
% TODO: Get canonical coordinates in to the text somehow
% TODO: Write about canonical forms of generators
