\chapter{Theory}

In this chapter, the mathematical theory used in the thesis will be presented.

%=============================================================================
\section{Point symmetries}

% TODO: Write symmetry introduction
Given a \dots

\dots

\begin{equation} \label{eq:first-order-ode}
  \dv{y}{x} = \omega(x,y)
  \implies
  \dv{\hat{y}}{\hat{x}} = \omega(\hat{x},\hat{y})
\end{equation}

\dots

The symmetry condition can then be formulated as: % FIXME: not proper wording
\begin{defn}[Point symmetry] \label{defn:first-order-symmetry}
  A transformation \(\Gamma: \pqty{x,y} \mapsto \pqty{\hat{x}(x,y), \hat{y}(x,y)}\) is a point symmetry of the ODE \ref{eq:first-order-ode} if
  \begin{equation}
    \dv{y}{x} = \omega(x,y)
    \implies
    \dv{\hat{y}}{\hat{x}} = \omega(\hat{x},\hat{y}).
  \end{equation}
\end{defn}
Using the total derivative and the definition of the ODE \ref{eq:first-order-ode},
\begin{equation}
  \dv{\hat{y}}{\hat{x}} = 
  \frac{\tD{x} \hat{y}(x,y)}{\tD{x} \hat{x}(x,y)} =
  \frac{\partial_x \hat{y} + y' \partial_y \hat{y}}{\partial_x \hat{x} + y' \partial_y \hat{x}} =
  \frac{\partial_x \hat{y} + \omega(x,y) \partial_y \hat{y}}{\partial_x \hat{x} + \omega(x,y) \partial_y \hat{x}},
\end{equation}
which means that \cref{defn:first-order-symmetry} can be written as
\begin{lem} \label{lem:simple-first-order-symmetry}
  A transformation \(\Gamma: \pqty{x,y} \mapsto \pqty{\hat{x}(x,y), \hat{y}(x,y)}\) is a point symmetry of the ODE \ref{eq:first-order-ode} if
  \begin{equation} \label{eq:simple-first-order-symmetry}
    \frac{\partial_x \hat{y} + \omega(x,y) \partial_y \hat{y}}{\partial_x \hat{x} + \omega(x,y) \partial_y \hat{x}} = \omega(\hat{x},\hat{y})
  \end{equation}
  holds.
\end{lem} % FIXME: Requirement of diffeomorphism
Given a transformation \(\Gamma\), it is therefore easy to check whether it is a point symmetry of the ODE \ref{eq:first-order-ode}.
The reverse process (finding point symmetries of \cref{eq:first-order-ode}) is on the other hand not easy, as \(\omega(\hat{x},\hat{y})\) will in all but the most trivial cases result in a non-linear PDE in with two unknown functions.
By restricting attention to a large subset of point symmetries these conditions can be simplified.

%=============================================================================
\section{Lie point symmetries}

A Lie group is a continuous group, where the continuity between members of the group can be described by one or more real valued parameters.
% TODO: Add more material on Lie groups, maybe as reference or in appendix.
For point symmetries, this means that a group of Lie point symmetries will transform a given point to a differentiable manifold, e.g. a line, plane, circle etc., for every point in the plane.
More simply, the transformations in a one-parameter Lie group can be written as
\begin{equation}
  \Gamma_\epsilon: \pqty{x,y} \mapsto \pqty{\hat{x}_\epsilon(x,y), \hat{y}_\epsilon(x,y)},
\end{equation}
where both \(\hat{x}_\epsilon(x,y)\) and \(\hat{y}_\epsilon(x,y)\) are differentiable in \(\epsilon\) when fixing \(x\) and \(y\).
This is most easily showed by a simple example.
\begin{exmp}
  The ODE
  \begin{equation} \label{eq:ex-a:ode}
    \dv{y}{x} = y
  \end{equation}
  has several symmetries.
  One group of symmetries is
  \begin{equation} \label{eq:ex-a:transformation}
    \pqty{\hat{x}_\epsilon(x,y), \hat{y}_\epsilon(x,y)} = \pqty{x + \epsilon, y}, \quad
    \forall \epsilon \in \reals
  \end{equation}
  The group consists of point symmetries of \cref{eq:ex-a:ode}, since
  \begin{equation}
    \frac{\partial_x \hat{y}_\epsilon + \omega(x,y) \partial_y \hat{y}_\epsilon}{\partial_x \hat{x}_\epsilon + \omega(x,y) \partial_y \hat{x}_\epsilon} =
    \frac{y}{1} =
    \hat{y} =
    \omega(\hat{x}_\epsilon,\hat{y}_\epsilon), \quad
    \forall \epsilon.
  \end{equation}
  So by \cref{defn:simple-first-order-symmetry} the transformations are point symmetries.
  The group is a Lie group, since by fixing \(\pqty{x,y}=\pqty{x^*\!,y^*}\),
  \begin{equation}
    \dv{\epsilon} \pqty{\hat{x}_\epsilon(x^*\!,y^*), \hat{y}_\epsilon(x^*\!,y^*)} =
    \pqty{\dv{\hat{x}(\epsilon)}{\epsilon}, \dv{\hat{y}(\epsilon)}{\epsilon}} =
    \pqty{1,0}
  \end{equation}
  exists for any \(\pqty{x^*\!,y^*}\).
\end{exmp}
Since Lie point symmetries act on a subset of the plane,
\begin{equation}
  \eval{\dv{\epsilon} \pqty{\hat{x}_\epsilon(x^*\!,y^*), \hat{y}_\epsilon(x^*\!,y^*)}}_{\epsilon=0} = \pqty{\xi(x^*\!,y^*), \eta(x^*\!,y^*)},
\end{equation}
where is \(\pqty{x^*\!,y^*}\) any point in the subset, is defined for that same subset. % FIXME: Figure out how to talk about inconsistencies (right now subset talk)
On that subset,
\begin{equation} \label{eq:tangent-field}
  \pqty{\xi(x,y), \eta(x,y)}
\end{equation}
can be viewed as a vector field characterizing the Lie group, commonly called the tangent field.
% TODO: Prove that the characterization is one-to-one
Note that the entire group consisting of mappings \(\Gamma_\epsilon\) is now represented without any reference to \(\epsilon\) in the tangent field \ref{eq:tangent-field}.

%-----------------------------------------------------------------------------
\subsection{The linearized symmetry condition}

By focusing on the tangent field characterization of one-parameter Lie symmetry groups, the symmetry condition \ref{lem:simple-first-order-symmetry} can be further simplified.
\begin{thm} \label{thm:linearized-first-order-symmetry}
  A Lie group of point transformations is a symmetry of the ODE \ref{eq:first-order-ode} if
  \begin{equation}
    \partial_x \eta + (\partial_y \eta - \partial_x \xi) \omega - \partial_y (\xi) \omega^2 =
    \xi \partial_x \omega + \eta \partial_y \omega,
  \end{equation}
  where \(\pqty{\xi(x,y), \eta(x,y)}\) is the tangent field of the Lie group.
\end{thm} % FIXME: Is this a correct formulation of the theorem?
To prove this, simply take the derivative of \cref{eq:simple-first-order-symmetry} in \(\epsilon\) and set \(\epsilon\) to 0.
For a more formal proof, refer to [thing]. % FIXME: Should have a reference

% TODO: Figure out better structure for this part
To simplify the notation, a differential operator
\begin{equation}
  X = \xi(x,y) \partial_x + \eta(x,y) \partial_y,
\end{equation}
called an infinitesimal generator, can be introduced.
As the generator is defined only by \(\xi\) and \(\eta\), it too serve as a characterization of the Lie symmetry group.
The infinitesimal generator can be prolonged... % How do I make this flow naturally?
