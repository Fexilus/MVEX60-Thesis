\addsec{Introduction}

In many scientific fields the concept of symmetry is important.
Symmetries bridge the gap between the qualitative and the quantitative; it is both a statement about what a system is, and how a system can be measured.
In most cases when symmetries are discussed, the symmetries in question are simple, geometric symmetries.
These can be mirror symmetries, as that of a face, or rotational symmetries, as that of a dice.
But from a mathematical point of view, the concept of symmetries is far broader.

This more general view of symmetries is used in physics to describe the fundamental properties of the universe.
These properties are called conservation laws, and are one of the many tools that have given physics its stable theoretical basis.
Establishing an equally solid theoretical framework for other fields, such as biochemistry and ecology, is of great interest.
In these fields the systems studied are of great complexity, just as in physics, but the design of experiments poses other challenges to those whishing to study them.
Since the studied subjects are often alive, distinct phenomena can not be entirely isolated when designing experiments.
While great progress has been made in the fields, the question still stands: which properties are inherent of these living systems, and which properties simply emerge in the modeling of the systems?

Symmetries could be one of the keys to answering this question.
However, the mathematics involved when studying these types of symmetries are involved, especially when applying them to systems unlike those found in physics, around which much of the mathematics has evolved.
This thesis therefore tries to chart some ground when it comes to applications of these methods to specific models.
Several models, stemming from both biochemistry and ecology, are studied.
The models vary in complexity, which allows the strength of the techniques to be displayed for simpler models, while still highlighting the obstacles of applying the methods to the complex models used at the edge of current research.

The common denominator for the models investigated in this thesis is that they are based on first order ordinary differential equations (ODE:s).
ODE:s are the most common type of models in biology \cite{maliksheriff2020biomodels,biomodels}, and first order ODE:s in particular are the norm.
These models are mathematically distinct from the higher order partial differential equation (PDE) models used in physics.
This means that the use of symmetries in the domain of biology is not merely a superficial question of applying known methods to problems from new fields.
Instead, even just the initial problem of finding symmetries brings up mathematical questions that are often disregarded in the literature, as first order models are viewed primarily as as a theoretical stepping stone to the higher order models that are traditionally studied.

This thesis focuses on the problem of finding symmetries of mathematical models in biology based on first order ODE:s.
Current literature on the computational aspects of symmetries of first order ODE:s is sparse, and mainly aims at finding a particular amount of symmetries \cite{chebterrab1997computer,chebterrab1998patterns}. % TODO: Maybe too few sources (both from same author)
This is due both to there always being an infinite amount of symmetries for first order ODE:s, and due to the intended use of the found symmetries.
While one purpose of finding symmetries is to fundamentally understand the studied systems by for example finding conservation laws, another common purpose is integrating problems that are hard to solve.
For first order ODE:s, the purpose in the literature almost always falls in the latter category while our interest lies in the former.
The focus of this thesis thus centers on finding all symmetries of a first order system given some restriction, so that biological properties can be systematically studied.

Further this thesis aims at framing questions of importance to more widespread adoption of similar techniques in biological fields.
These involve both biologically interpreting symmetries, and understanding what the symmetries can be used for.
The discussion centers around the studied models, using them as examples to get a better grasp of larger questions.
A special focus is put on the Gompertz model, the simplest of the models not previously studied using symmetries.

In chapter 1, a general overview of the use of first order ODE:s as models in biology is given.
The four models studied in this thesis are also introduced: the Hill equation, the Gompertz model, the Lotka-Volterra predator--prey model and the Yildirim--Mackey lactose operon model.

In chapter 2, the mathematical theory of Lie point symmetries is introduced.
The text is aimed at being accessible to readers who have no prior experience of mathematical symmetries.
The first three sections presents and explains the tools necessary for finding Lie point symmetries of systems of first order ODE:s.
The sections should act as sufficient background to understand the calculations performed in chapter 3 and 4, given that the reader has a prior understanding of the fundamentals of ODE:s.
The last section concerns the underlying algebraic structure of Lie point symmetries.
It is significantly more mathematically involved than previous sections out of necessity, and a reader inexperienced in abstract algebra might want to put off reading this section until after chapter 3 and 4.

In chapter 3, symmetries are calculated for the four models presented in chapter 1 using the standard method for first order ODE:s: ansätze.
The chapter mostly consists of traditional symmetry calculations, and can thus safely be skimmed if the reader is already familiar with Lie point symmetries.

Chapter 4 contains calculations of symmetries for the latter three models using a method novel in this setting.
The method is based on parameter independence, and is repurposed from the subfield of group classification.
The method of parameter independence is more systematic than using ansätze, and thus offers a distinct alternative approach to finding symmetries of systems of first order ODE:s.

In chapter 5, different ways of interpreting and using symmetries are discussed in brief.
Two such uses are explored further: invariants and the general structure of symmetries.
In particular, different formulations of the Gompertz model are compared by looking at the general structure of their respective symmetries.

The thesis ends on a discussion in chapter 6, where the results of are reviewed and put into a more general context.
The discussion is aimed at being as non-mathematical as possible, so that readers with any mathematical background can get a feeling for the role symmetries could play in mathematical modeling in biology.
