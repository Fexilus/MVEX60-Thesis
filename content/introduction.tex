\addsec{Introduction}

In many scientific fields the concept of symmetry is important.
Symmetries bridge the gap between the qualitative and the quantitative; it is both a statement about what a system is, and how a system can be measured.
In most cases when symmetries are discussed, the symmetries in question are simple, geometric symmetries.
These can be mirror symmetries, as that of a face, or rotational symmetries, as that of a dice.
But from a mathematical point of view, the concept of symmetries is far broader.

This more general view of symmetries is used in physics to describe the fundamental properties of the universe.
These properties are called conservation laws, and are one of the many tools that have given physics it's stable theoretical basis.
Finding equally stable bases for other fields, such as biochemistry and ecology, are of great interest.
In these fields the systems studied are of great complexity, just as in physics, but the design of experiments pose other challenges to those whishing to study them.
Since the studied subjects are often alive, distinct phenomena can not be entirely isolated when designing experiments.
While great progress has been made in the fields, the question still stands: which properties are inherent of these living systems, and which properties simply emerge in the modeling of the systems?

Symmetries could be one of the keys to answering this question.
However, the mathematics involved when calculating these types of symmetries are involved, especially when applying them to systems unlike those found in physics.
This thesis will therefore try to chart some ground when it comes to applications of these methods to specific models.
Several models, stemming from both biochemistry and ecology, will be studied.

In this half-way thesis, two such models have been studied.
Firstly the Hill model, used for modeling reaction kinetics in cells, is studied.
This model has been previously studied using symmetry methods, but is given a thorough treatment not previously published.
Secondly the Gompertz model, used to model both animal and tumor growth, is studied.
The model is usually only used in its explicit form, so symmetry methods can be used to shine light on the difference between different differential forms found in literature.
