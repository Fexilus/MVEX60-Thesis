\addsec{Introduction}

In many scientific fields the concept of symmetry is important.
Symmetries bridge the gap between the qualitative and the quantitative; it is both a statement about what a system is, and how a system can be measured.
In most cases when symmetries are discussed, the symmetries in question are simple, geometric symmetries.
These can be mirror symmetries, as that of a face, or rotational symmetries, as that of a dice.
But from a mathematical point of view, the concept of symmetries is far broader.

This more general view of symmetries is used in physics to describe the fundamental properties of the universe.
These properties are called conservation laws, and are one of the many tools that have given physics it's stable theoretical basis.
Finding equally stable bases for other fields, such as biochemistry and ecology, are of great interest.
In these fields the systems studied are of great complexity, just as in physics, but the design of experiments pose other challenges to those whishing to study them.
Since the studied subjects are often alive, distinct phenomena can not be entirely isolated when designing experiments.
While great progress has been made in the fields, the question still stands: which properties are inherent of these living systems, and which properties simply emerge in the modeling of the systems?

Symmetries could be one of the keys to answering this question.
However, the mathematics involved when calculating these types of symmetries are involved, especially when applying them to systems unlike those found in physics.
This thesis therefore tries to chart some ground when it comes to applications of these methods to specific models.
Several models, stemming from both biochemistry and ecology, are studied.
The models vary in complexity, which allows the strength of the techniques to be displayed for simpler models, while still highlighting the obstacles with applying the methods to the complex models used at the edge of current research.

One thing that all of the models have in common is that they are first order ordinary differential equations (ODE:s).
First order ODE:s are one of the most common type of models in biology, distinct from the higher order models used in physics.
This means that the use of symmetries in the domain of biology is not merely a superficial question of applying known methods to problems from new fields.
Instead, even just the initial problem of finding symmetries brings up mathematical questions that are often disregarded in literature, as first order models are often seen as a theoretical stepping stone to the higher order models that are traditionally studied.

Firstly, this thesis focuses on the problem of finding symmetries of first order ODE:s.
While there is some literature on the subject, the aim of that research is often to find a particular amount of symmetries.
This is due both to there always being an infinite amount of symmetries for first order ODE:s, and due to the intended use of the found symmetries.
While one purpose of finding symmetries is fundamentally understanding systems and finding conservation laws, another common purpose is integrating problems that are hard to solve.
For first order ODE:s, the purpose in literature almost always falls in the latter category while our interest lies in the former.
The discussion of this thesis thus centers around finding all symmetries of a first order system given some restriction, so that biological properties can be systematically found.

Secondly, this thesis tries to frame questions of importance to more widespread adoption of similar techniques in biological fields.
These involve both biologically interpreting symmetries, and understanding what the symmetries can be used for.
The discussing centers around the studied models, using them as examples to get a better grasp of larger questions.
Special focus is put on the Gompertz model, the simplest of the models not previously studied using symmetries.
