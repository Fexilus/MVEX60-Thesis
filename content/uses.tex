\chapter{The interpretation and structure of symmetries in biology} \label{ch:uses}

In the previous two chapters, two different methods were used to find symmetries of parametrized first order ODE:s.
However, finding symmetries has little to no value unless they can be used to better understand biological processes.
While some of the possible uses for symmetries have been mentioned in passing in earlier chapters, this chapter intends to explore these ideas more fully.
The chapter consists of two parts.
In the first introductory part, an overview of how symmetries could fit into biological modeling is presented.
Most ideas are open ended, as they constitute one or several research project on their own.
In the second part, one such use, namely the interpretation of symmetries as relating to invariants of the differential system, is explored for the biological models for which symmetries have been found.
It turns out, based on the mathematics in \cref{sec:lie-point-properties}, that the relation between invariants of the differential system and the symmetries thereof is quite straight forward for first order ODE systems.

\section{Symmetries as a tool in modeling} \label{sec:symmetries-as-tool}

Three of the larger obstacles when constructing biological models are model construction, model validation and model selection.
There is potential for symmetries to be used to address all these three areas of modeling.

Model construction is the process of going from biological knowledge to a mathematical model.
In this process, models are often built from first principles and are simplified by making additional assumptions, also known as model reduction.
If a proper theory of symmetries in biology were to exist, symmetries could potentially be used to enhance this process.
The selection of appropriate first principles could be enhanced by knowing invariant and symmetric properties of those first principles.
Additionally, model reduction could be better understood during model construction by looking at which symmetries are preserved during reduction and which are broken.
These and similar methods could essentially act as a form of pre-screening for the model validation.

Model validation is the process of comparing the mathematical model developed in model construction to experimental data.
In the case of ODE models such as those studies in this thesis, the available data will be several time series of measurements of the system studied.
Even for non-parametrized ODE models, the process of obtaining predictive estimates from the model is not trivial as non-linear systems of ODE:s do not generally have analytic solutions.
Instead, a numerical estimation must be performed, which depending on the ODE system might be computationally costly.
To further complicate matters, such simulations are not possible to perform for general parameterized models, as concrete parameter values are needed.
Instead, the parameters must be considered statistically at the same time as the model is simulated.
This process is substantially more computationally costly, and must be reperformed every time the model is changed.
As symmetries can be used to partially or completely solve differential equations, it would be of interest to study if this could be used to reduce the computations involved in this process.
Symmetries could also be used as another form of pre-screening, comparing symmetries of the model to symmetries of the data.
This hinges on the calculations of symmetries being significantly less computationally costly than those of parameter estimation.

If that is not the case, symmetries could still be used during model selection, where several models that all fit available data must be selected amongst.
More costly symmetry calculations could in this step be used to compare model fit to data transformed with symmetries, thus differentiating between models where traditional methods fail \cite{ohlsson2020symmetry}.
Common to all of the applications of symmetries after the fact that data is introduced is the question of how parameters should be treated.
As previously mentioned there is an uncertainty around the parameters that is taken into account when evaluating models.
Symmetries as treated in this thesis ignore all probabilistic properties of both the models themselves and of parameters.
To properly apply symmetries in these contexts, more general theory has to be used to take such properties into account.

This also raises the question of whether parameters in symmetry generators are desirable or not.
When using and seeing symmetries as relating to fundamental biological concepts, symmetry generators containing few or no parameters could be considered more useful, as fewer parameters of the model has to be fixed for a generator to relate to concrete transformations.
Such symmetries would then correspond to more general biological concepts.
An example of this is the symmetry generated by \(\partial_t\), corresponding to time invariance.
As the symmetry is not dependent on any parameter, a common interpretation can be made for all models: the symmetry signals that the system evolves independently of factors dependent on time.
For every parameter in a generator, more assumptions on the model in question is needed; a meaningful interpretation of the symmetry can not be made until an interpretation of the parameters exist.

On the other hand, parameters in symmetry generators is what would make differentiation between similar models in model selection possible.
A productive way to think about these concepts is to view parameterized models as sets of models.
The question of whether parameters in symmetry generators are desirable or not then simplifies to the question of whether the use of the symmetry is to differentiate between subsets of this set of models, or to describe properties of the entire set of models.

% Using symmetry with param. ind. but Ovsiyannikov-like to get symmetries independent of arbitrary functions as parameters. Could then single cell variance be found in data looking at symmetries?


\section{Invariant solutions under symmetry of the Hill equation}

% TODO: This is a different kind of invariance, so a connection is needed
To better understand the underlying source of the Lie symmetry groups of the Hill equation, it is of interest to find curves invariant under the solution.
The reduced characteristic of the generator
\begin{equation*}
  X_2 = - \left( (n-1) x + n y \right) \partial_\tau + y \partial_y
\end{equation*}
is
\begin{align*}
  \bar{Q} &= 
  \eta - \omega \xi = 
  y - \left( -\frac{y^n}{1+y^2} \right) \left( - \left( (n-1) \tau + n y \right) \right) =\\
  &= y \left(1 -\frac{y^{n-1} \left( (n-1) \tau + n y \right)}{1+y^2} \right).
\end{align*}
Since solutions are invariant under a symmetry if the characteristic is 0 on the entire solution curve, invariant solutions must satisfy
\begin{equation*}
  y \left(1 -\frac{y^{n-1} \left( (n-1) \tau + n y \right)}{1+y^n} \right) = 0.
\end{equation*}
The solutions that meet this condition (aside from the trivial case \(y \equiv 0\)) must thus be on the form
\begin{equation*}
  y^{n-1} \left( (n-1) \tau + n y \right) = 1+y^n,
\end{equation*}
which is more clearly stated as
\begin{equation*}
  (n-1) \left( \tau y^{n-1} + y^n \right) = 1.
\end{equation*}
For all \(n>0\) except \(n=1\) such solutions exist, and take the form
\begin{equation} \label{eq:hill-invariants}
  \tau y^{n-1} + y^n = \frac{1}{n-1}.
\end{equation}
Additionally, since \(\tau\) is (dimensionless) time, any application of the model will have an initial condition on the form \(y(0) = y_0\).
By fixing \(\tau=0\) in \cref{eq:hill-invariants}, the initial condition of an invariant curve given any \(n>0, n\neq1\) is shown to be
\begin{equation*}
  y(0) =\frac{1}{(n-1)^{1/n}}.
\end{equation*}

\section{The structure of the symmetry groups of the Gompertz models}

As discussed in \cref{sec:lie-point-properties}, all symmetries of first order systems of ODE:s can be separated in to two parts: the trivial component and the characteristic component.
Furthermore, the characteristic component's general form depends only on the invariants of the differential system.
Here this connection will be shown explicitly for all formulations of the Gompertz model, serving both as a case study and as a further investigation of the different formulations of the Gompertz model.

\subsection{The classical Gompertz model}

The symmetries of the classical Gompertz model found using an ansatz in \cref{ch:ansatze} were
\begin{align*}
  X_{\text{c},1} &= e^{k_{G} t} \partial_t \\
  X_{\text{c},2} &= W e^{k_{G} t + e^{T_{i} k_{G}} e^{- k_{G} t}} \partial_t \\
  X_{\text{c},3} &= e^{- e^{T_{i} k_{G}} e^{- k_{G} t}} \partial_W \\
  X_{\text{c},f} &= \frac{f{\left(t \right)} e^{- T_{i} k_{G}} e^{k_{G} t}}{k_{G}} \partial_t + W f{\left(t \right)} \partial_W
\end{align*}
and the symmetries found using parameter independence in \cref{ch:param-ind} were
\begin{align*}
  X_{\text{c},1} &= e^{k_G t} \partial_t \\
  X_{\text{c},4} &= W \partial_W \\
  X_{\text{c},5} &= \partial_t - k_G W \ln\left(W\right) \partial_W.
\end{align*}
The corresponding reduced characteristics of the symmetry generators are
\begin{align*}
  \bar{Q}_{\text{c},1} &= -k_G e^{k_G T_i} W \\
  \bar{Q}_{\text{c},2} &= - k_G e^{k_{G} T_i + e^{-k_G (t - T_i)}} W^2 \\
  \bar{Q}_{\text{c},3} &= e^{- e^{- k_{G} \left(t - T_{i}\right)}} \\
  \bar{Q}_{\text{c},4} &= W \\
  \bar{Q}_{\text{c},5} &= - k_G W \ln\left(W\right) - k_G e^{-k_G (t - T_i)} W \\
  \bar{Q}_{\text{c},f} &= 0.
\end{align*}
Thus it is clear that the symmetry generator \(X_{\text{c},f}\) is a trivial symmetry generator.
From the mathematical theory the reduced characteristic of any symmetry generator can be written as
\begin{equation*}
  \bar{Q} = (\partial_W J)^{-1} I,
\end{equation*}
where \(J\) consists of functionally independent invariants and \(I\) is any invariant.
As the classical Gompertz model is a scalar first order ODE, the \(J\) is one-dimensional, and thus the quotient of two reduced characteristics must be an invariant.
Picking the reduced characteristics \(\bar{Q}_{\text{c},4}\) and \(\bar{Q}_{\text{c},5}\) that differ by more than a constant expression (assuming the parameters are constant),
\begin{equation*}
  \frac{\bar{Q}_{\text{c},5}}{\bar{Q}_{\text{c},4}} = - k_G \left(\ln\left(W\right) + e^{-k_G (t - T_i)} \right) = I
\end{equation*}
is an invariant since
\begin{equation*}
  X_{\text{c}, \text{T}}(I) = I_t + k_G e^{-k_G (t - T_i)} W I_W = k_G^2 e^{-k_G (t - T_i)} - k_G e^{-k_G (t - T_i)} W \frac{k_G}{W} = 0.
\end{equation*}
For the particular case of scalar first order equations with known solutions such as the classical Gompertz ODE, the above described methodology is not the most direct path to finding such an invariant.
The formulation
\begin{equation*}
  I_t + k_G e^{-k_G (t - T_i)} W I_W = 0
\end{equation*}
of the invariance condition can be solved using the method of characteristics, which reformulates the problem as
\begin{equation*}
  \frac{\dl{t}}{1} = \frac{\dl{W}}{k_G e^{-k_G (t - T_i)} W}
\end{equation*}
which simplifies to
\begin{equation*}
  \dl \left(e^{-k_G (t - T_i)} + \ln\left(W\right)\right) = 0
\end{equation*}
leading to the equivalent invariant.
Returning to the task of determining the general structure of the group of all symmetries,
\begin{equation*}
  J(t, W) = e^{-k_G (t - T_i)} + \ln\left(W\right)
\end{equation*}
can be taken.
Thus
\begin{equation*}
  \partial_W J = \frac{1}{W}
\end{equation*}
and
\begin{equation*}
  \left(\partial_W J\right)^{-1} = W.
\end{equation*}
Similarly, an arbitrary invariant \(I\) of the differential equation can be written as
\begin{equation*}
  I(t, W) = F\left(e^{-k_G (t - T_i)} + \ln\left(W\right)\right)
\end{equation*}
for and arbitrary function \(F\).
Thus, the reduced characteristic of any symmetry generator must have the form
\begin{equation} \label{eq:general-classical-gompertz-characteristic}
  \bar{Q} = W F\left(e^{-k_G (t - T_i)} + \ln\left(W\right)\right)
\end{equation}
and the general form for a symmetry generator is
\begin{equation} \label{eq:general-classical-gompertz-symmetry}
  X = \xi(t, W) \partial_t + W \left(k_G e^{-k_G (t - T_i)} \xi(t, W) + F\left(e^{-k_G (t - T_i)} + \ln\left(W\right)\right)\right) \partial_W
\end{equation}
for arbitrary functions \(\xi\) and \(F\).

\subsection{The autonomous Gompertz model}

The symmetries of the autonomous Gompertz model found using an ansatz in \cref{ch:ansatze} were
\begin{align*}
  X_{\text{a},1} &= e^{k_G t} \ln\left(\frac{W}{A}\right) \partial_t \\
  X_{\text{a},2} &= e^{-k_G t} W \partial_W \\
  X_{\text{a},3} &= \ln\left(\frac{W}{A}\right) W \partial_W \\
  X_{\text{a},f} &= f(t) \partial_t - k_G f(t) \ln\left(\frac{W}{A}\right) W \partial_W
\end{align*}
and the symmetries found using parameter independence in \cref{ch:param-ind} were
\begin{align*}
  X_{\text{a},2} &= e^{-k_G t} W \partial_W \\
  X_{\text{a},3} &= \ln\left(\frac{W}{A}\right) W \partial_W \\
  X_{\text{a},4} &= \partial_t \\
  X_{\text{a},5} &= t \partial_t + \ln\left(\abs{\ln\left(\frac{W}{A}\right)}\right) \ln\left(\frac{W}{A}\right) W \partial_W \\
  X_{\text{a},6} &= e^{-k_G t} \partial_t - k_G e^{-k_G t} \ln\mathopen{}\left(W\right)\mathclose{} W \partial_W.
\end{align*}
The corresponding reduced characteristics of the symmetry generators are
\begin{align*}
  \bar{Q}_{\text{c},1} &= k_G e^{k_G t} \left(\ln\left(\frac{W}{A}\right)\right)^2 W \\
  \bar{Q}_{\text{c},2} &= e^{-k_G t} W \\
  \bar{Q}_{\text{c},3} &= \ln\left(\frac{W}{A}\right) W\\
  \bar{Q}_{\text{c},4} &= k_G \ln\left(\frac{W}{A}\right) W \\
  \bar{Q}_{\text{c},5} &= \ln\left(\abs{\ln\left(\frac{W}{A}\right)}\right) \ln\left(\frac{W}{A}\right) W + k_G t \ln\left(\frac{W}{A}\right) W \\
  \bar{Q}_{\text{c},6} &= - k_G \ln\left(A\right) e^{-k_G t} W \\
  \bar{Q}_{\text{c},f} &= 0.
\end{align*}
Just as for the classical Gompertz model, there are several ways to find a function \(J\) consisting of functionally independent invariants of the autonomous Gompertz model.
Using the most straight-forward method,
\begin{equation*}
  \frac{\bar{Q}_{\text{c},5}}{\bar{Q}_{\text{c},3}} = \ln\left(\abs{\ln\left(\frac{W}{A}\right)}\right) + k_G t
\end{equation*}
is an invariant, and thus
\begin{equation*}
  J(t, W) = \ln\left(\abs{\ln\left(\frac{W}{A}\right)}\right) + k_G t
\end{equation*}
can be chosen.
Thus
\begin{equation*}
  \partial_W J = \frac{1}{\ln\left(\frac{W}{A}\right) W}
\end{equation*}
and
\begin{equation*}
  \left(\partial_W J\right)^{-1} = \ln\left(\frac{W}{A}\right) W.
\end{equation*}
Using \cref{cor:reduced-characteristic-decomposition}, the reduced characteristic of symmetry generators of the autonomous Gompertz model thus has the general form
\begin{equation} \label{eq:general-autonomous-gompertz-characteristic}
  \bar{Q} = \ln\left(\frac{W}{A}\right) W F\left(\ln\left(\abs{\ln\left(\frac{W}{A}\right)}\right) + k_G t\right),
\end{equation}
where \(F\) is an arbitrary function, with the general form for a symmetry generator being
\begin{equation} \label{eq:general-autonomous-gompertz-symmetry}
  X = \xi(t, W) \partial_t + \ln\left(\frac{W}{A}\right) W \left(-k_G \xi(t, W) + F\left(\ln\left(\abs{\ln\left(\frac{W}{A}\right)}\right) + k_G t\right)\right) \partial_W
\end{equation}
for arbitrary functions \(\xi\) and \(F\).

\subsection{The system Gompertz model}

The symmetries of the system Gompertz model found using an ansatz in \cref{ch:ansatze} were
\begin{align*}
  X_{\text{s},1} &= \partial_t \\
  X_{\text{s},2} &= - \frac{e^{k_{G} t}}{k_{G}} \partial_t + G e^{k_{G} t} \partial_G \\
  X_{\text{s},3} &= e^{k_{G} t} G \partial_t \\
  X_{\text{s},4} &= W \partial_W \\
  X_{\text{s},5} &= - \frac{W e^{- k_{G} t}}{k_{G}} \partial_W + e^{- k_{G} t} \partial_G
\end{align*}
and the symmetries found using parameter independence in \cref{ch:param-ind} were
\begin{align*}
  X_{\text{s},1} &= \partial_t \\
  X_{\text{s},4} &= W \partial_W \\
  X_{\text{s},6} &= \ln\left(W\right) W \partial_W + G \partial_G.
\end{align*}
The corresponding reduced characteristics of the symmetry generators are
\begin{align*}
  \bar{\vect{Q}}_{\text{s},1} &= \left(-W G, k_G G\right) \\
  \bar{\vect{Q}}_{\text{s},2} &= \left(\frac{e^{k_{G} t}}{k_{G}} W G, 0\right) \\
  \bar{\vect{Q}}_{\text{s},3} &= \left(- e^{k_{G} t} W G^2, k_G e^{k_{G} t} G^2\right) \\
  \bar{\vect{Q}}_{\text{s},4} &= \left(W, 0\right) \\
  \bar{\vect{Q}}_{\text{s},5} &= \left(- \frac{W e^{- k_{G} t}}{k_{G}}, e^{- k_{G} t} \right) \\
  \bar{\vect{Q}}_{\text{s},6} &= \left(\ln\left(W\right) W, G\right).
\end{align*}
Since the system Gompertz model has two equations, two functionally independent invariants exist.
While it is a bit more cumbersome than in the scalar case to calculate invariants from the reduced characteristics, it is still possible.
However, in the particular case of the system Gompertz ODE, using the method of characteristics to solve the invariance condition is also a viable approach.
Both methods will be shown here to exemplify how the calculations are made for non-scalar ODE:s.

Using the reduced characteristics, one invariant can be found using the fact that
\begin{equation*}
  \bar{\vect{Q}}_{\text{s},3} = e^{k_{G} t} G \left(-W G, k_G G\right) = e^{k_{G} t} G \bar{\vect{Q}}_{\text{s},1}
\end{equation*}
implies that \(e^{k_{G} t} G\) is an invariant.
Using this invariant, all of the reduced characteristics \(\bar{\vect{Q}}_{\text{s},1}, \dots, \bar{\vect{Q}}_{\text{s},5}\) found using the ansatz method can be divided into two sets of vectors that only differ by invariant coefficients.
\begin{align*}
  \bar{\vect{Q}}_{\text{s},1} &= \frac{\bar{\vect{Q}}_{\text{s},3}}{e^{k_{G} t} G} = \left(k_G e^{k_{G} t} G\right) \bar{\vect{Q}}_{\text{s},5} \\
  \frac{e^{k_{G} t} G}{k_G} \bar{\vect{Q}}_{\text{s},2} &= \bar{\vect{Q}}_{\text{s},4}.
\end{align*}
Since no function \(f\) could result in the relationship \(-\bar{\vect{Q}}_{\text{s},1} = f(t, W, G) bar{\vect{Q}}_{\text{s},4}\),
\begin{equation*}
  \begin{pmatrix}
    W & W G \\
    0 & -k_G G
  \end{pmatrix}
  = \left(\partial_{(W, G)} \vect{J}\right)^{-1}
\end{equation*}
must hold for some function \(J\) consisting of functionally independent invariants.
Thus a valid \(J\) is the solution to the differential equation
\begin{equation} \label{eq:universal-invariant-system-gompertz-hint}
  \partial_{(W, G)} \vect{J} =
  \begin{pmatrix}
    W & W G \\
    0 & -k_G G
  \end{pmatrix}^{-1}
\end{equation}
that also fulfills the original invariance condition
\begin{equation*}
  \partial_{(t, W, G)} \vect{J} \cdot
  \begin{pmatrix}
    1 \\
    W G \\
    -k_G G
  \end{pmatrix}
  = 0.
\end{equation*}
Thus, finding a \(J\) consisting of functionally independent invariants using only the reduced characteristics \(\bar{\vect{Q}}_{\text{s},1}, \dots, \bar{\vect{Q}}_{\text{s},5}\) found using the ansatz method is essentially the problem of finding invariants, but with \cref{eq:universal-invariant-system-gompertz-hint} serving as a hint.
But, using the reduced characteristic \(\bar{\vect{Q}}_{\text{s},6}\) found using the parameter independence method, this calculation can be avoided since
\begin{multline*}
  \bar{\vect{Q}}_{\text{s},6} = \left(\ln\left(W\right) W, G\right) = \left(-\frac{W G}{k_G}, G\right) + \left(\ln\left(W\right) W + \frac{W G}{k_G}, 0\right) =\\= \frac{1}{k_G} \bar{\vect{Q}}_{\text{s},1} + \left(\ln\left(W\right) + \frac{G}{k_G}\right) \bar{\vect{Q}}_{\text{s},4}.
\end{multline*}
\(\ln\left(W\right) + \frac{G}{k_G}\) must thus be an invariant, and since \(e^{k_{G} t} G\) and \(\ln\left(W\right) + \frac{G}{k_G}\) are functionally independent,
\begin{equation*}
  \vect{J}(t, W, G) = \left(\ln\left(W\right) + \frac{G}{k_G}, \ln\left(e^{k_{G} t} G\right)\right) = \left(\ln\left(W\right) + \frac{G}{k_G}, k_G t + \ln\left(G\right)\right)
\end{equation*}
must consist of functionally independent invariants.

Using the method of characteristics to solve the invariance condition
\begin{equation*}
  I_t + W G I_W - k_G G I_G = 0,
\end{equation*}
results in the reformulated problem
\begin{equation*}
  \frac{\dl t}{1} = \frac{\dl W}{W G} = \frac{\dl G}{- k_G G}.
\end{equation*}
The expression
\begin{equation*}
  \frac{\dl W}{W G} = \frac{\dl G}{- k_G G}
\end{equation*}
simplifies to
\begin{equation*}
  \dl \left(\ln\left(W\right) + \frac{G}{k_G}\right) = 0,
\end{equation*}
while
\begin{equation*}
  \frac{\dl t}{1} = \frac{\dl G}{- k_G G}
\end{equation*}
simplifies to
\begin{equation*}
  \dl \left(k_G t + \ln\left(G\right)\right) = 0.
\end{equation*}
Thus
\begin{equation} \label{eq:universal-invariant-system-gompertz}
  \vect{J}(t, W, G) = \left(\ln\left(W\right) + \frac{G}{k_G}, k_G t + \ln\left(G\right)\right),
\end{equation}
consisting of functionally independent invariants, is once again found.

Using the \(J\) in \labelcref{eq:universal-invariant-system-gompertz},
\begin{equation*}
  \partial_{(W, G)} \vect{J} =
  \begin{pmatrix}
    \frac{1}{W} & \frac{1}{k_G} \\
    0 & \frac{1}{G}
  \end{pmatrix}
\end{equation*}
and thus
\begin{equation*}
  \left(\partial_{(W, G)} \vect{J}\right)^{-1} =
  \begin{pmatrix}
    W & - \frac{W G}{k_G} \\
    0 & G
  \end{pmatrix}.
\end{equation*}
Using \cref{cor:reduced-characteristic-decomposition}, the reduced characteristic of any symmetry generator must thus have the form
\begin{equation} \label{eq:general-system-gompertz-characteristic}
  \bar{\vect{Q}} =
  \begin{pmatrix}
    W & - \frac{W G}{k_G} \\
    0 & G
  \end{pmatrix}
  \cdot
  \begin{pmatrix}
    F_1\left(\ln\left(W\right) + \frac{G}{k_G}, k_G t + \ln\left(G\right)\right) \\
    F_2\left(\ln\left(W\right) + \frac{G}{k_G}, k_G t + \ln\left(G\right)\right)
  \end{pmatrix},
\end{equation}
where \(F_1\) and \(F_2\) are arbitrary functions.
The general form of Lie point symmetry generators of the system Gompertz model is thus
\begin{align*}
  X =& \xi(t, W, G) \partial_t + \\
  &+ \left(W G \xi(t, W, G) + W I_1(t, W, G) - \frac{W G}{k_G} I_2(t, W, G)\right) \partial_W + \\
  &+ \left(-k_G G \xi(t, W, G) + G I_2(t, W, G)\right) \partial_G,
\end{align*}
where \(\xi\) is an arbitrary function and
\begin{align*}
  I_1(t, W, G) &= F_1\left(\ln\left(W\right) + \frac{G}{k_G}, k_G t + \ln\left(G\right)\right) \\
  I_2(t, W, G) &= F_2\left(\ln\left(W\right) + \frac{G}{k_G}, k_G t + \ln\left(G\right)\right)
\end{align*}
are invariants.

\subsection{Comparison of the Gompertz models}

With the general forms of the Gompertz model Lie point symmetries established, further insight to the relationship between the three models can be gained.
Since both the classical and the autonomous model exist in the same space (\(B \simeq \reals(t)\) and \(F \simeq \reals(W)\)), comparison between the scalar models is a natural starting point.

The trivial symmetry generator
\begin{equation*}
  X_{\text{a},T} = \partial_t - k_G \ln\left(\frac{W}{A}\right) W \partial_W
\end{equation*}
of the autonomous Gompertz model is symmetry of the classical Gompertz model, since
\begin{equation*}
  -k_G \ln\left(\frac{W}{A}\right) W = W \left(k_G e^{-k_G (t - T_i)} - k_G \left(e^{-k_G (t - T_i)} + \ln\left(W\right)\right) + k_G \ln\left(A\right)\right),
\end{equation*}
and thus \(X_{a,T}\) can be written on the general form \cref{eq:general-classical-gompertz-symmetry} of classical Gompertz symmetries, with \(F(x) = -k_G x + k_G \ln\left(A\right)\) and \(\xi(t, W) \equiv 1\).
Similarly, the trivial symmetry generator
\begin{equation*}
  X_{\text{c},T} = \partial_t + k_G e^{-k_G (t - T_i)} W \partial_W
\end{equation*}
of the classical Gompertz model can be written on the general form \cref{eq:general-autonomous-gompertz-symmetry} of autonomous Gompertz symmetries, since
\begin{equation*}
  k_G e^{-k_G (t - T_i)} W = \ln\left(\frac{W}{A}\right) W \left(-k_G + k_G e^{k_G T_i} e^{-\left(\ln\left(\abs{\ln\left(\frac{W}{A}\right)}\right) + k_G t\right)} + k_G\right). % TODO: Figure out abs problems
\end{equation*}
The parameter \(k_G\) is assumed to have the same value in both models for these statements to hold, which is reasonable since the parametrization of all three models is done in such a way that \(k_G\) has the same effect on solution curves.
It is also worth noting that the parameters \(A\) and \(T_i\) are arbitrary values for the classical and autonomous Gompertz models respectively; they have no relation to the properties they represent when they are not part of the ODE formulation.
As long as \(A\) is considered an invariant of the classical Gompertz model and \(T_i\) is considered an invariant of the autonomous Gompertz model, the arbitrary invariant multipliers of the characteristic components \(X_{\text{c}, \bar{\vect{Q}}}\) and \(X_{\text{a}, \bar{\vect{Q}}}\) of the respective symmetries can be chosen so that the vector field corresponding to the autonomous model is a symmetry of the classical model and vice versa.

The system Gompertz model can not as stringently be compared to the scalar models.
However, by using some intuition the connection between the models and their symmetries can be shown to be quite straight forward.
Consider first the classical model and the system model.
In this case, \(G\) is \enquote{equal} to \(k_G e^{-k_G (t - T_i)}\).
The general form of the reduced characteristic for the classical model \labelcref{eq:general-classical-gompertz-characteristic} can then be seen as the first column and first invariant in the general form of the reduced characteristic for the system model \labelcref{eq:general-system-gompertz-characteristic}, since
\begin{equation*}
  \bar{Q}_{\text{c}} = W F\left(e^{-k_G (t - T_i)} + \ln\left(W\right)\right)
\end{equation*}
then has the same form as
\begin{equation} \label{eq:gompertz-chacteristic-system-classic-similarity}
  \bar{\vect{Q}}_{\text{s}} =
  \begin{pmatrix}
    W \\
    0
  \end{pmatrix}
  \cdot F_1\left(\ln\left(W\right) + \frac{G}{k_G}\right).
\end{equation}

Instead considering the autonomous and the system model, \(G\) is \enquote{equal} to \(-k_G \ln\left(\frac{W}{A}\right) W\)
The general form of the reduced characteristic for the autonomous model \labelcref{eq:general-autonomous-gompertz-characteristic} can then be seen as the second column and second invariant in the general form of the reduced characteristic for the system model \labelcref{eq:general-system-gompertz-characteristic}, since
\begin{equation*}
  \bar{Q}_{\text{a}} = \ln\left(\frac{W}{A}\right) W F\left(\ln\left(\abs{\ln\left(\frac{W}{A}\right)}\right) + k_G t\right)
\end{equation*}
then has the same form as
\begin{equation} \label{eq:gompertz-chacteristic-system-autonomous-similarity}
  \bar{\vect{Q}}_{\text{s}} =
  \begin{pmatrix}
    - \frac{W G}{k_G} \\
    G
  \end{pmatrix}
  \cdot F_2\left(k_G t + \ln\left(G\right)\right).
\end{equation}

Together, \cref{eq:gompertz-chacteristic-system-classic-similarity} and \cref{eq:gompertz-chacteristic-system-autonomous-similarity} make up a subset of the general form in \cref{eq:general-system-gompertz-characteristic} of the reduced characteristic for the system Gompertz model.
The two functions \(F_1\) and \(F_2\) only depend on one invariant each in the comparison to the symmetries of the scalar models, as opposed to both invariants in the general form for the system Gompertz model.
As such, the symmetries of the system model capture more than the sum of the symmetries of the two scalar models.

\section{The structure of the symmetry group of the Lotka--Volterra model}

The symmetries of the Lotka--Volterra model found using an ansatz in \cref{ch:ansatze} were
\begin{align*}
  X_1 &= \partial_t \\
  X_2 &= \frac{-bNP + aN}{c} \partial_N + \frac{cNP - dP}{c} \partial_P \\
  X_3 &= \frac{t}{c} \partial_t + \frac{-btNP + atN}{c} \partial_N + \frac{ctNP - dtP}{c} \partial_P \\
  X_4 &= \frac{N}{c} \partial_t + \frac{-bcN^2P + acN^2 - bdNP + adN}{c^2} \partial_N + \frac{c^2N^2P - d^2P}{c^2} \partial_P \\
  X_5 &= \frac{P}{c} \partial_t + \frac{-bNP^2 + aNP}{c} \partial_N + \frac{cNP^2 - dP^2}{c} \partial_P
\end{align*}
and the symmetries found using parameter independence in \cref{ch:param-ind} were
\begin{equation*}
  X_1 = \partial_t.
\end{equation*}
The corresponding reduced characteristics of the symmetry generators are
\begin{align*}
  \bar{\vect{Q}}_1 &= - \left(a N - b N P, c N P - d P\right) \\
  \bar{\vect{Q}}_2 &= \frac{1}{c} \left(a N - b N P, c N P - d P\right) \\
  \bar{\vect{Q}}_3 &= 0 \\
  \bar{\vect{Q}}_4 &= \frac{d}{c^2}\left(aN- bNP, cNP - dP\right) \\
  \bar{\vect{Q}}_5 &= 0.
\end{align*}
Thus only one of the potentially two forms of reduced characteristic (modulo multiplication by invariants) is found.
Furthermore, the reduced characteristics reveal no invariants by themselves.

Using the method of characteristics, one such invariant can be found with ease.
The invariance condition 
\begin{equation*}
  I_t + \left(a N - b N P\right) I_N + \left(c N P - d P\right) I_P = 0
\end{equation*}
results in the reformulated problem
\begin{equation} \label{eq:moc-lotka-volterra}
  \frac{\dl t}{1} = \frac{\dl N}{a N - b N P} = \frac{\dl P}{c N P - d P}.
\end{equation}
The second equality can be simplified to
\begin{equation*}
  \frac{\dl N}{N} \left(c N - d\right) = \frac{\dl P}{P} \left(a - b P\right)
\end{equation*}
which has the solution
\begin{equation} \label{eq:invariant-lotka-volterra}
  \dl \left(cN + bP - d \ln\left(N\right) - a \ln\left(P\right)\right) = 0.
\end{equation}
The first equality in \cref{eq:moc-lotka-volterra} does not have an explicit solution, for the same reason the Lotka--Volterra system itself does not have an explicit solution, but must instead be studied mostly using the invariant in \cref{eq:invariant-lotka-volterra} \cite{murray2002biology}.

Using the characteristics and invariants easily calculable, it can thus be concluded that all reduced characteristics
\begin{equation*} %\label{eq:lotka-volterra-characteristic}
  \bar{\vect{Q}} = F\left(cN + bP - d \ln\left(N\right) - a \ln\left(P\right)\right) \left(a N - b N P, c N P - d P\right)
\end{equation*}
correspond to symmetry generators of the Lotka--Volterra predator prey model \labelcref{eq:lotka-volterra}, where \(F\) is an arbitrary function.
All generators on the form
\begin{align*}
  X =& \xi(t, N, P) \partial_t + \\
  &+ \left(a N - b N P\right) \left(\xi(t, N, P) + F\left(cN + bP - d \ln\left(N\right) - a \ln\left(P\right)\right)\right) \partial_W + \\
  &+ \left(c N P - d P\right) \left(\xi(t, N, P) + F\left(cN + bP - d \ln\left(N\right) - a \ln\left(P\right)\right)\right) \partial_W
\end{align*}
are thus symmetry generators of the Lotka--Volterra model.
