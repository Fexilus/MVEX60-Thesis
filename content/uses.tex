\chapter{Uses of symmetries in biology}

In this chapter biological interpretations and uses of the found symmetries will be discussed\dots

\section{Invariant solutions under symmetry of the Hill model}

To better understand the underlying source of the Lie symmetry groups of the Hill equation, it is of interest to find curves invariant under the solution.
That is, solutions to the ODE that are mapped by the Lie symmetry group to the same curve.

% TODO: Add some references to theory and glue the paragraphs together
The reduced characteristic of the generator \(X_2 = - \left( (n-1) x + n y \right) \partial_\tau + y \partial_y\) is
\begin{align}
  \bar{Q} &= 
  \eta - \omega \xi = 
  y - \left( -\frac{y^n}{1+y^2} \right) \left( - \left( (n-1) \tau + n y \right) \right) =\\
  &= y \left(1 -\frac{y^{n-1} \left( (n-1) \tau + n y \right)}{1+y^2} \right).
\end{align}
Since solutions are invariant under a symmetry if the characteristic is 0 on the entire solution curve, invariant solutions must satisfy
\begin{equation}
  y \left(1 -\frac{y^{n-1} \left( (n-1) \tau + n y \right)}{1+y^n} \right) = 0.
\end{equation}
The solutions that meet this condition (aside from the trivial case \(y \equiv 0\)) must thus be on the form
\begin{equation}
  y^{n-1} \left( (n-1) \tau + n y \right) = 1+y^n,
\end{equation}
which is more clearly stated as
\begin{equation}
  (n-1) \left( \tau y^{n-1} + y^n \right) = 1.
\end{equation}
For all \(n>0\) except \(n=1\) such solutions exist, and take the form
\begin{equation} \label{eq:hill-invariants}
  \tau y^{n-1} + y^n = \frac{1}{n-1}.
\end{equation}
Determining the functions \(y(\tau)\) that are invariant for a given \(n\) can be bothersome, but some general insights can be made.
Since \cref{eq:hill-invariants} is a polynomial in \(y\) of degree \(n\), there will exist a finite amount of solutions (this holds for non-integer \(n\) too, as it can be bounded by Descartes' rule of signs). % FIXME: Strengthen this claim with a source https://math.stackexchange.com/a/1291211
Thus, the symmetries are not trivial since there exist an infinite amount of solutions to \cref{eq:hill}.
Additionally, since \(\tau\) is (dimensionless) time, any application of the model will have an initial condition on the form \(y(0) = y_0\).
By fixing \(\tau=0\) in \cref{eq:hill-invariants}, the initial condition of an invariant curve given any \(n>0, n\neq1\) is shown to be
\begin{equation}
  y(0) =\frac{1}{(n-1)^{1/n}}.
\end{equation}
