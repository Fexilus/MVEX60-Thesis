\chapter{Discussion}

Looking at the non-trivial symmetries of the autonomous Gompertz model found by using ansätze
\begin{align}
  X_1 &= e^{k_G t} \ln(\frac{W}{A}) \partial_t \\
  X_2 &= e^{-k_G t} W \partial_W \\
  X_3 &= \ln(\frac{W}{A}) W \partial_W
\end{align}
compared to the non-trivial symmetries found using the parameter independence method
\begin{align}
  X_{\text{a},1} &= \partial_t \\
  X_{\text{a},2} &= t \partial_t + \ln(\ln(\frac{W}{A})) \ln(\frac{W}{A}) W \partial_W\\
  X_{\text{a},3} &= \ln(\frac{W}{A}) W \partial_W \\
  X_{\text{a},4} &= e^{-k_G t} \partial_t - k_G e^{-k_G t} \ln(W) W \partial_W\\
  X_{\text{a},6} &= e^{-k_G t} W \partial_W,
\end{align}
it is clear that the parameter independence method is useful.
% Using symmetry with param. ind. but Ovsiyannikov-like to get symmetries independent of arbitrary functions as parameters. Could then single cell variance be found in data looking at symmetries?
