\chapter{Discussion}

As outlined in \cref{sec:symmetries-as-tool}, the possibilities for using symmetries in biological modeling is promising.
If using symmetries turns out to be viable in even one of the cases presented, or another way relevant to researchers in biology, problems that so far have been impossible or too time consuming to solve might suddenly be viable to approach.
The strength of using symmetries is that they relate to qualitative information about the systems studied, while most of the mathematical tools available to researchers at the moment relate to quantitative information.
Symmetries could therefore bridge the gap between established qualitative facts on the biological side of research and the mathematical models and simulations of those biological systems.

Common to all the possible uses for symmetries is the need for the researcher to be able to find symmetries of a system.
In this thesis only first order ODE:s were considered, and even in this case the calculations involved in finding symmetries grew to the scale were automation is needed, as seen in \cref{ch:ansatze,ch:param-ind}.
For most biological systems, ODE-models are simplifications of more sophisticated models that involve spatiality (and with that often higher order dynamics), randomness and time-delays.
While the mathematics involved in finding symmetries for these more sophisticated models is far more complex than the theory covered in this thesis, the fact that the concept of symmetries can be generalized to most settings is a cause for optimism.
The methods developed for determination of symmetries of ODE:s will thus have a good chance of being generalizable or at least have parallels when finding symmetries of more advanced systems.

The method of parameter independence developed in this thesis in \cref{ch:param-ind} solves problems in finding symmetries of first order ODE:s arising from the fact that the Lie algebras of the symmetries of first order ODE:s being infinite dimensional.
But even for higher order differential equations, especially PDE:s, the problem of finding a general form for symmetry generators might not be viable to solve.
The method of parameter independence could then be used to find general forms of at least all generators independent of some parameter in the same way as for first order ODE:s, as the method is compatible with the generalizations of the symmetry theory to higher order PDE:s.
Similar results should be expected for many methods developed for some specialized subset of differential equations, which means that the study symmetries of simplified biological models can act as a stepping stone to more robust methods.
This indicates that there are good conditions for further research on symmetries in biological modelling.
