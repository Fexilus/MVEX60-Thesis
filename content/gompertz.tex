\chapter{Gompertz model}

In this chapter the Gompertz model will be studied using Lie point symmetries.

The Gompertz model is common across several life science fields.
It is used to describe systems with initial exponential growth, that eventually asymptotically approaches some maximum size.
It has successfully been used to model\dots

%=============================================================================
\section{ODE description}

The Gompertz model has a clear analytical description, which is sufficient as description when modeling phenomena.
The form of the function describing the growth takes many forms, often varying depending on field and individual taste.
\cite{tjorve2017gompertz} shows that all useful formulations can be sorted into two groups, \(T_i\)- and \(W_0\)-formulations, with canonical forms
\begin{equation}
  \dv{W}{t} = A e^{-e^{k_G(t-T_i)}}
\end{equation}
and
\begin{equation}
  \dv{W}{t} = A \left(\frac{W_0}{A}\right)^{e^{-k_G t}}
\end{equation}
respectively.
In these equations, and for the remainder of the chapter, \(W(t)\) represents a measure of size dependent on time.
The sufficiency of these two formulations stems from the interpretation of the four parameters \(A\), \(k_G\), \(T_i\) and \(W_0\):
\(A\) is the value of the upper asymptote, also known as the carrying capacity of the system.
\(k_G\), although lacking in interpretation itself, is proportional to \(k_U = k_G / e\), where \(k_U\) is the relative (to \(A\)) maximum slope during the process.
Together, \(A\) and \(k_G\) control the shape of the curve.
\(T_i\) or \(W_0\) depending on the formulation control the localization of the curve.
\(T_i\) is the point in time where the curve achieves its maximum slope, also known as the point of inflection.
\(W_0\) on the other hand is the size at \(t=0\).
