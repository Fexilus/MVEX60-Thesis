\chapter{The Gompertz model}

In several fields in and outside of the life sciences growth plays an important role in general.
In particular, when measuring different phenomena ranging from cell growth to the growth of cities, the concept of exponential growth often appears.
Exponential growth stems from a species growing in proportion to the size of the species.
Written in mathematical terms
\begin{equation} \label{eq:exponential}
  \dv{W}{t} = c W(t),
\end{equation}
where \(W(t)\) is the size of our species (volume of a tumor, weight of an animal, individuals in a population etc.) at time \(t\) and \(c\) is a constant.
While exponential growth accurately models initially growth for many systems, eventually some external factor will limit the growth.
The external factor might be simple like limited availability of food in the case of an animal population, or complex like limitations of infrastructure in the case of cities. % FIXME: Check validity of statement
Correctly modeling the external limitations is crucial when studying the long term behavior of the system.

The Gompertz model is one such model that has seen success in many areas.
It was first proposed in 1825 as a means of predicting the mortality rate of populations in order to accurately prize life insurances and annuities \cite{gompertz1825nature}.
Gompertz formulated his model as the differential equation
\begin{equation} \label{eq:original-gompertz}
  a L_x \times q^x \cdot \dot{x} = - (L_x)\dot{}
\end{equation}
where \(L_x\) is the number living at age \(x\), or formulated in modern notation
\begin{equation} \label{eq:original-gompertz-modern}
  \dv{L}{x} = -a q^x L(x)
\end{equation}
where \(L(x)\) is the number of people living at age \(x\).
It is worth noting that at small \(x\) and \(c = -a\), \cref{eq:original-gompertz-modern} behaves like \cref{eq:exponential}.
In retrospect this similarity is not surprising as Gompertz modeled decay (of a population), which can be seen as \enquote{negative growth}.
This connection was however at first not used.

Around a hundred years after the models conception it saw its first use as a growth model, modeling economic growth \cite{prescott1922demand,peabody1924railway}.
It was first mentioned the life sciences in 1926 as a suggestion of an alternative growth model in a review \cite{wright1926reviews}, and a few years later saw its first concrete use modeling the weight of cattle \cite{davidson1928growth}.
In the century since, the Gompertz model has been used to model the size of a wide range of animals (for a good summary, see \cite{tjorve2017gompertz}).
The breadth of animals (and parts of animals) where the Gompertz model can be fitted well to growth data has made this one of the life sciences where the model is most used.
The other branch of the life sciences where the Gompertz model has seen success is in modeling tumor growth.
The model was first used (apart as a tool for making graphs \cite{casey1934alteration}) for this purpose in 1964 \cite{laird1964dynamics}.
It has since become one of the most widely used models for tumor growth \cite{gerlee2013muddle} (for a summary of applications, see the introduction of \cite{benzekry2014classical}).
When studying both organism and tumor growth, the same question have been asked about the Gompertz model, namely what the biological interpretation of the model is.
In this chapter, this question will be tackled using the theory of Lie point symmetries.

%=============================================================================
\section{Finding a standardized ODE description}

Even though Gompertz first stated his model as the ODE \ref{eq:original-gompertz}, the differential form is not the most commonly used.
Instead the Gompertz model is usually formulated as the solution to \cref{eq:original-gompertz-modern}.
In Gompertz original paper, this function takes the form
\begin{equation} \label{eq:original-gompertz-solution}
  L(x) = d g^{q^x}
\end{equation}
where \(g = \exp(-a/\ln(q))\) in \cref{eq:original-gompertz-modern}.
Note that the parameter \(d\) is not included in the differential equation but instead stems from the constant of integration.
By replacing Gompertz' \(L(x)\) (amount of people of age \(x\)) with \(W(t)\) (the size of a species at time \(t\)) the function can be used to model growth without any structural changes to the function.
The function form of the model is not only sufficient as description when the goal is to fit the model to some data; the third parameter \(d\) is necessary since it relates to the initial value needed to solve \cref{eq:original-gompertz-modern}.
\Cref{eq:original-gompertz-solution} is however not a form where Lie point symmetries can be employed.
Additionally, the function is not parametrized consistently across literature, varying depending on field and taste.
To analyze the model with Lie point symmetries it is therefore necessary to determine two things.
Firstly it must be established what is meant by \enquote{The Gompertz Model} when viewing the model as an ODE through a life since lens.
Secondly, a standardized and meaningful parametrization of the model in ODE form must be established in order to gain insight from the later Lie point symmetry treatment.

Since the function form of the model is sufficient and necessary to match data, the ODE form is only found in literature as a means of providing a background to the model.
The ODE formulation thus often lacks proper references, rendering the origin of the formulation hard to trace.
All formulations of the ODE found in literature can however be sorted into one of three families.
The first family can be written on the form
\begin{equation} \label{eq:rough-gompertz-classical-ode}
  \dv{W}{t} = \alpha e^{-b t} W(t).
\end{equation}
These ODE:s are reparameterizations of Gompertz' original ODE \ref{eq:original-gompertz-modern}, often emphasizing the expected behaviors of the model in the growth context by choosing parameters that should be positive.
In the parametrization seen in \cref{eq:rough-gompertz-classical-ode}, \(\alpha\) should be positive for the species size \(W(t)\) to grow (as opposed to shrinking or decaying) and \(b\) should be positive for the growth to reduce over time (and thus limiting the growth).
This family of ODE:s is tightly related to the second family, which can be written on the form
\begin{align}
  \dv{W}{t} &= \gamma(t) W(t)
  \dv{\gamma}{t} &= -b \gamma(t).
\end{align}

In an attempt to standardize the formulations, \cite{tjorve2017gompertz} (concerned mainly with the growth of organisms) shows that all \enquote{useful} formulations can be sorted into two groups: \(T_i\)- and \(W_0\)-formulations.
These have canonical forms
\begin{equation} \label{eq:gompertz-ti-function}
  W(t) = A e^{-e^{-k_G(t-T_i)}}
\end{equation}
and
\begin{equation} \label{eq:gompertz-w0-function}
  W(t) = A \left(\frac{W_0}{A}\right)^{-e^{-k_G t}}
\end{equation}
respectively.
% TODO: Make it clear that these are just separate parametrizations
The sufficiency of these two formulations stems from the interpretation of the four parameters \(A\), \(k_G\), \(T_i\) and \(W_0\):
\(A\) is the value of the upper asymptote, also known as the carrying capacity of the system.
\(k_G\), although lacking in interpretation itself, is proportional to \(k_U = k_G / e\), where \(k_U\) is the relative (to \(A\)) maximum slope during the process.
Together, \(A\) and \(k_G\) control the shape of the curve.
\(T_i\) or \(W_0\) depending on the formulation control the localization of the curve.
\(T_i\) is the point in time where the curve achieves its maximum slope, also known as the point of inflection.
\(W_0\) on the other hand is the size at \(t=0\).

For the purpose of finding symmetries, this description on function-form is insufficient.
Instead, an ODE-formulation is desired.
Several different ODEs have \cref{eq:gompertz-ti-function,eq:gompertz-w0-function} as a solution.
The ODEs found in literature can be rewritten on either the form
\begin{equation} \label{eq:gompertz-og-autonomous-ode}
  \dv{W}{t} = -\alpha W \ln(\frac{W}{K})
\end{equation}
as seen in \cite{bajzer1997tumor,delauro2014stochastic}, or the form
\begin{equation} \label{eq:gompertz-og-nonautonomous-ode}
  \dv{W}{t} = r e^{-b t} W
\end{equation}
as seen in \cite{burger2019epidemic}.
\Cref{eq:gompertz-og-autonomous-ode} is an autonomous ODE, while \cref{eq:gompertz-og-nonautonomous-ode} is non-autonomous.
This fact alone shows that the formulations are fundamentally different.
To further interpret this difference, the ODEs can be rewritten using the parameters in \cref{eq:gompertz-ti-function,eq:gompertz-w0-function}.
\Cref{eq:gompertz-og-autonomous-ode,eq:gompertz-og-nonautonomous-ode} have solutions on the forms
\begin{equation} \label{eq:gompertz-autonomous}
  W(t) = K e^{c e^{-\alpha t}}
\end{equation}
and
\begin{equation} \label{eq:gompertz-nonautonomous}
  W(t) = c e^{-r/b \cdot e^{-b t}}
\end{equation}
respectively, where \(c\) is an integration constant.
Comparing the general solutions in \cref{eq:gompertz-autonomous,eq:gompertz-nonautonomous} to the \(T_i\)- and \(W_0\)-formulations in \cref{eq:gompertz-ti-function,eq:gompertz-w0-function}, we find that the ODEs \ref{eq:gompertz-og-autonomous-ode} and \ref{eq:gompertz-og-nonautonomous-ode} can be rewritten on the forms:
\begin{flalign}
    \text{Autonomous, } & T_i \text{ and } W_0 :& W' &= -k_G W \ln(\frac{W}{A}) &&\FlLabel{eq:gompertz-autonomous-ode} \\
    \text{Non-autonomous, } & T_i :& W' &= k_G e^{-k_G (t - T_i)} W &&\FlLabel{eq:gompertz-nonautonomous-ti-ode} \\
    \text{Non-autonomous, } & W_0 :& W' &= k_G \ln(\frac{W_0}{A})e^{-k_G t} W &&\FlLabel{eq:gompertz-nonautonomous-w0-ode}
\end{flalign}


%=============================================================================
\section{Finding Lie point symmetries}
\begin{align}
  X_1 &= e^{k_G t} \ln(\frac{W}{A}) \partial_t \\
  X_2 &= e^{-k_G t} W \partial_W \\
  X_3 &= W \ln(\frac{W}{A}) \partial_W \\
  X_F &= F(t) \partial_t - k_G F(t) W \ln(\frac{W}{A}) \partial_W
\end{align}
are Lie point symmetries of \cref{eq:gompertz-autonomous-ode}.
