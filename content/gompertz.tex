

%=============================================================================


%=============================================================================
\section{Future leads}

Some future leads when studying the Gompertz model using symmetry methods are:
\begin{itemize}
  \item Try proving allometry properties when considering embryogenesis of the Gompertz model \cite{deakin1970allometry}.
  \item Reevaluate motivations of the Gompertz model \cite{bajzer1997tumor}.
  \item See if a more advanced empirical model that also hold in small time shares any symmetries \cite{frenzen1986justification}.
  \item See if a necrosis model shares any symmetries \cite{milotti2012interplay}.
\end{itemize}

%=============================================================================
\section{Half way summary}

At the current point in time, a few conclusions can be drawn.
Firstly, without analyzing the symmetries, some of the differences of the two ODE formulations of the Gompertz model are apparent.
The classical ODE concerns itself with growth systems with specific timings, while the autonomous ODE concerns itself with growth systems with specific carrying capacities.
In and of it self this is of course valuable information.
But secondly, the symmetry calculations show that the autonomous ODE has distinct symmetries.
If conclusions can be drawn about the connection of one of those symmetries to a qualitative property of a growth system, this could be used as a tool to invalidate the hypothesis that the classical ODE correctly models a phenomena.
Conversely, if distinct symmetries of the classical ODE can be found and connected to qualitative properties, those could be used to invalidate the use of the autonomous ODE for some problems.
