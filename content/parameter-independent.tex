\chapter{Finding symmetries using parameter independence}

Here there will be an introduction on the parameter independence method, mentioning that it is inspired by \cite{ovsiannikov1982group} etc.

\section{The Gompertz model}
To reiterate, the Gompertz models are\par\noindent
\begin{tabularx}{\linewidth}{rrM}
  Classical, & \(T_i\) :&
  \begin{minipage}{\linewidth}
    \begin{equation}
      \dv{W}{t} = k_G e^{-k_G (t - T_i)} W(t) \label{eq:gompertz-classical-ti}
    \end{equation}
  \end{minipage}\tabularnewline
  Classical, & \(W_0\) :&
  \begin{minipage}{\linewidth}
    \begin{equation}
      \dv{W}{t} = k_G \ln(\frac{W_0}{A})e^{-k_G t} W(t) \label{eq:gompertz-classical-w0}
    \end{equation}
  \end{minipage}\tabularnewline
  Autonomous, & \(T_i\) and \(W_0\) :&
  \begin{minipage}{\linewidth}
    \begin{equation}
      \dv{W}{t} = -k_G \ln(\frac{W(t)}{A}) W(t) \label{eq:gompertz-autonomous}
    \end{equation}
  \end{minipage}\tabularnewline
  System, & \(T_i\) and \(W_0\) :&
  \begin{minipage}{\linewidth}%
    {\begin{subequations}\label{eq:gompertz-system}
      \begin{align}
        \dv{W}{t} &= G(t) W(t) \label{eq:gompertz-system-a}\\
        \dv{G}{t} &= -k_G G(t). \label{eq:gompertz-system-b}
      \end{align}
    \end{subequations}}%
  \end{minipage}
\end{tabularx}

\subsection{The classical Gompertz model}

For the classical Gompertz model, two different parametrizations exist.
For the purposes of the parameter independence method, either of these parametrizations may be used since the parameters \(T_i\) and \(W_0\) only depend on each other and the parameter \(k_G\), but neither the time \(t\) nor the state \(W\).
The \(T_i\) will be used in these calculations.

The linearized symmetry condition \labelcref{eq:linearized-first-order-symmetry} is for the classical \(T_i\)-parameterized Gompertz model
\begin{equation}\label{eq:gompertz-classical-lin-symmetry-cond}
  \begin{split}
    \eta_t &+ k_G e^{-k_G (t - T_i)} W\left(\eta_W - \xi_t\right) - (k_G)^2 e^{-2k_G (t - T_i)} W^2 \xi_W +\\ &+ (k_G)^2 e^{-k_G (t - T_i)} W \xi - k_G e^{-k_G (t - T_i)} \eta = 0.
  \end{split}
\end{equation}
For a generator \(\qty(\xi, \eta)\) to be independent of a parameter, \cref{eq:gompertz-classical-lin-symmetry-cond} is decomposed by functionally independent coefficients of that parameter.

\subsubsection{\texorpdfstring{\(k_G\)-independent symmetries}{Growth rate-independent symmetries}}

Decomposition of \cref{eq:gompertz-classical-lin-symmetry-cond} in \(k_G\) gives the parameter independence determining equations
\begin{subequations}
  \begin{flalign}
    1 & : & \eta_t &= 0 && \FlLabel{eq:gompertz-classical-det-kg-a}\\
    k_G e^{-k_G (t - T_i)} & : & W \left(\eta_W - \xi_t\right) - \eta &= 0 && \FlLabel{eq:gompertz-classical-det-kg-b}\\
    (k_G)^2 e^{-k_G (t - T_i)} & : & W \xi &= 0 && \FlLabel{eq:gompertz-classical-det-kg-c}\\
    (k_G)^2 e^{-2k_G (t - T_i)} & : & W^2 \xi_W &= 0. && \FlLabel{eq:gompertz-classical-det-kg-d}
  \end{flalign}
\end{subequations}
From \cref{eq:gompertz-classical-det-kg-c} it is clear that
\begin{equation}
  \xi \equiv 0,
\end{equation}
and hence \cref{eq:gompertz-classical-det-kg-d} must also hold.
From \cref{eq:gompertz-classical-det-kg-a} it is clear that
\begin{equation}
  \eta = \eta(W)
\end{equation}
is a function only in \(W\).
This leaves \cref{eq:gompertz-classical-det-kg-b} that simplifies into the ODE
\begin{equation}
  W \eta_W - \eta = 0
\end{equation}
with the general solution
\begin{equation}
  \eta = c_1 W,
\end{equation}
where \(c_1\) is an arbitrary constant.
Thus any \(k_G\)-independent symmetry generator of the classical Gompertz model \labelcref{eq:gompertz-classical-ti} must have the form
\begin{align}
  \xi &= 0 \\
  \eta &= c_1 W,
\end{align}
which is spanned by the generator basis
\begin{equation}
  X_{\text{c},1} = W \partial_W.
\end{equation}

\subsubsection{\texorpdfstring{\(T_i\)-independent symmetries}{Inflection time-independent symmetries}}

Decomposition of \cref{eq:gompertz-classical-lin-symmetry-cond} in \(T_i\) gives the parameter independence determining equations
\begin{subequations}
  \begin{flalign}
    1 & : & \eta_t &= 0 && \FlLabel{eq:gompertz-classical-det-ti-a}\\
    e^{k_G T_i} & : & k_G e^{-k_G t} W \left(\eta_W - \xi_t\right) + (k_G)^2 e^{-k_G t} W \xi - k_G e^{-k_G t} \eta &= 0 && \FlLabel{eq:gompertz-classical-det-ti-b}\\
    e^{2 k_G T_i} & : & -(k_G)^2 e^{-2 k_G t} W^2 \xi_W &= 0 && \FlLabel{eq:gompertz-classical-det-ti-c}
  \end{flalign}
\end{subequations}
From \cref{eq:gompertz-classical-det-ti-a,eq:gompertz-classical-det-ti-c} it is clear that
\begin{align}
  \eta &= \eta(W) \\
  \xi = \xi(t)
\end{align}
respectively.
Dividing \cref{eq:gompertz-classical-det-ti-b} by \(k_G e^{-k_G t}\) shows that
\begin{equation}\label{eq:gompertz-classical-det-ti-b-simple}
  W \left(\eta_W - \xi_t\right) + k_G W \xi - \eta = 0.
\end{equation}
Since \(\xi\) (and thus its derivative) is the only source of time dependence in \cref{eq:gompertz-classical-det-ti-b-simple},
\begin{equation}
  k_G W \xi - W \xi_t = f(W).
\end{equation}
for some function \(f\) in \(W\).
And since \(\xi\) (and thus its derivative) are functions only in \(t\)
\begin{equation}
  k_G \xi - \xi_t = c_1,
\end{equation}
where \(c_1\) is an arbitrary constant, which in turn has the general solution
\begin{equation}
  \xi = c_2 e^{k_G t} + \frac{c_1}{k_G}
\end{equation}
with an additional arbitrary constant \(c_2\).
Insertion of this result in \cref{eq:gompertz-classical-det-ti-b-simple} yields
\begin{equation}
  W \eta_W - \eta + c_1 W = 0,
\end{equation}
which, since \(\eta\) is only a function in \(W\), is a straight forward ODE with the general solution
\begin{equation}
  \eta = -c_1 W \ln(W) + c_3 W.
\end{equation}
Thus any \(T_i\)-independent symmetry generator of the classical Gompertz model \labelcref{eq:gompertz-classical-ti} must have the form
\begin{align}
  \xi &= \frac{c_1}{k_G} + c_2 e^{k_G t} \\
  \eta &= -c_1 W \ln(W) + c_3 W,
\end{align}
which is spanned by the generator basis
\begin{align}
  X_{\text{c},2} &= \partial_t - k_G W \ln(W) \partial_W \\
  X_{\text{c},3} &= e^{k_G t} \partial_t \\
  X_{\text{c},4} &= W \partial_W = X_{\text{c},1}.
\end{align}

\subsubsection{Analysis of the found generators}
The basis of all the generators found by the parameter independence method for the classical Gompertz model are
\begin{align}
  X_{\text{c},1} &= W \partial_W \\
  X_{\text{c},2} &= \partial_t - k_G W \ln(W) \partial_W \\
  X_{\text{c},3} &= e^{k_G t} \partial_t.
\end{align}
By looking at the Lie brackets
\begin{align}
  \comm{X_{\text{c},1}}{X_{\text{c},2}} &= -k_G X_{\text{c},1} \\
  \comm{X_{\text{c},1}}{X_{\text{c},3}} &= 0 \\
  \comm{X_{\text{c},2}}{X_{\text{c},3}} &= k_G X_{\text{c},3}.
\end{align}
it is clear that they constitute a full Lie algebra.


\subsection{The autonomous Gompertz model}

For the autonomous Gompertz model there is only one parametrization of interest.
The linearized symmetry condition \labelcref{eq:linearized-first-order-symmetry} is for the autonomous Gompertz model
\begin{equation}\label{eq:gompertz-autonomous-lin-symmetry-cond}
  \begin{split}
    \eta_t &- k_G \ln(\frac{W}{A}) W\left(\eta_W - \xi_t\right) - (k_G)^2 \left(\ln(\frac{W(t)}{A})\right)^2 W^2 \xi_W +\\ &+ k_G \left(\ln(\frac{W}{A}) + 1\right) \eta = 0.
  \end{split}
\end{equation}

\subsubsection{\texorpdfstring{\(k_G\)-independent symmetries}{Growth rate-independent symmetries}}

Decomposition of \cref{eq:gompertz-autonomous-lin-symmetry-cond} in \(k_G\) gives the parameter independence determining equations
\begin{subequations}
  \begin{flalign}
    1 & : & \eta_t &= 0 && \FlLabel{eq:gompertz-autonomous-det-kg-a}\\
    k_G & : & -\ln(\frac{W}{A}) W\left(\eta_W - \xi_t\right) + \left(\ln(\frac{W}{A}) + 1\right) \eta &= 0 && \FlLabel{eq:gompertz-autonomous-det-kg-b}\\
    (k_G)^2 & : & -\left(\ln(\frac{W(t)}{A})\right)^2 W^2 \xi_W &= 0. && \FlLabel{eq:gompertz-autonomous-det-kg-c}
  \end{flalign}
\end{subequations}
From \cref{eq:gompertz-autonomous-det-kg-a,eq:gompertz-autonomous-det-kg-c}
\begin{align}
  \eta &= \eta(W) \\
  \xi = \xi(t).
\end{align}
Hence, since the only source of time dependence in \cref{eq:gompertz-autonomous-det-kg-b} is \(\xi_t\),
\begin{equation}
  \xi_t = c_1
\end{equation}
where \(c_1\) is an arbitrary constant, and thus
\begin{equation}
  \xi = c_2 t + c_1
\end{equation}
for an additional arbitrary constant \(c_2\).
\Cref{eq:gompertz-autonomous-det-kg-b} can thus be rewritten as
\begin{equation}
  \eta_W - \frac{\ln(\frac{W}{A}) + 1}{\ln(\frac{W}{A}) W} \eta - c_1 = 0,
\end{equation}
which is a scalar first order ODE with the solution
\begin{equation}
  \eta = c_1 \ln(\ln(\frac{W}{A})) \ln(\frac{W}{A}) W + c_3 \ln(\frac{W}{A}) W.
\end{equation}
Thus any \(k_G\)-independent symmetry generator of the autonomous Gompertz model \labelcref{eq:gompertz-autonomous} must have the form
\begin{align}
  \xi &= c_2 + c_1 t \\
  \eta &= c_1 \ln(\ln(\frac{W}{A})) \ln(\frac{W}{A}) W + c_3 \ln(\frac{W}{A}) W,
\end{align}
which is spanned by the generator basis
\begin{align}
  X_{\text{a},1} &= \partial_t \\
  X_{\text{a},2} &= t \partial_t + \ln(\ln(\frac{W}{A})) \ln(\frac{W}{A}) W \partial_W\\
  X_{\text{a},3} &= \ln(\frac{W}{A}) W \partial_W.
\end{align}

\subsubsection{\texorpdfstring{\(A\)-independent symmetries}{Carrying capacity-independent symmetries}}

Decomposition of \cref{eq:gompertz-autonomous-lin-symmetry-cond} in \(A\) gives the parameter independence determining equations
\begin{subequations}
  \begin{flalign}
    1 & : & \eta_t + k_G \eta &= 0 && \FlLabel{eq:gompertz-autonomous-det-a-a}\\
    \ln(\frac{W}{A}) & : & - k_G W\left(\eta_W - \xi_t\right)  + k_G \eta &= 0 && \FlLabel{eq:gompertz-autonomous-det-a-b}\\
    \left(\ln(\frac{W}{A})\right)^2 & : & - (k_G)^2 W^2 \xi_W &= 0. && \FlLabel{eq:gompertz-autonomous-det-a-c}
  \end{flalign}
\end{subequations}
From \cref{eq:gompertz-autonomous-det-a-c}
\begin{equation}
  \xi = \xi(t).
\end{equation}
Since \(\eta\) (and thus its derivative) are the only unknown sources of \(W\)-dependence in \cref{eq:gompertz-autonomous-det-a-b},
\begin{equation}
  \eta_W - \frac{1}{W}\eta = \xi_t(t)
\end{equation}
must hold.
Integration in \(W\) gives
\begin{equation}
  \eta = \ln(W) W \xi_t(t) + W f(t)
\end{equation}
for some arbitrary function \(f\) in time.
Inserting this result in \cref{eq:gompertz-autonomous-det-a-a} gives
\begin{equation}
  \ln(W) W \xi_{tt} + W f_t + k_G \ln(W) W \xi_t + k_G W f
\end{equation}
which can be decomposed by \(W\) into
\begin{align}
  \xi_{tt} + k_G \xi_t &= 0 \\
  f_t + k_G f &= 0
\end{align}
with solutions
\begin{align}
  \xi &= - c_1 \frac{1}{k_G} e^{-k_G t} + c_2 \\
  f &= c_3 e^{-k_G t}.
\end{align}
Thus any \(A\)-independent symmetry generator of the autonomous Gompertz model \labelcref{eq:gompertz-autonomous} must have the form
\begin{align}
  \xi &= - c_1 \frac{1}{k_G} e^{-k_G t} + c_2 \\
  \eta &= c_1 e^{-k_G t} \ln(W) W  + c_3 e^{-k_G t} W,
\end{align}
which is spanned by the generator basis
\begin{align}
  X_{\text{a},4} &= e^{-k_G t} \partial_t - k_G e^{-k_G t} \ln(W) W \partial_W\\
  X_{\text{a},5} &= \partial_t = X_{\text{a},1}\\
  X_{\text{a},6} &= e^{-k_G t} W \partial_W.
\end{align}

\subsubsection{Analysis of the found generators}
The basis of all the generators found by the parameter independence method for the autonomous Gompertz model are
\begin{align}
  X_{\text{a},1} &= \partial_t \\
  X_{\text{a},2} &= t \partial_t + \ln(\ln(\frac{W}{A})) \ln(\frac{W}{A}) W \partial_W\\
  X_{\text{a},3} &= \ln(\frac{W}{A}) W \partial_W \\
  X_{\text{a},4} &= e^{-k_G t} \partial_t - k_G e^{-k_G t} \ln(W) W \partial_W\\
  X_{\text{a},6} &= e^{-k_G t} W \partial_W.
\end{align}
Looking at the Lie brackets
\begin{align}
  \comm{X_{\text{a},1}}{X_{\text{a},2}} &= X_{\text{a},1} \\
  \comm{X_{\text{a},1}}{X_{\text{a},3}} &= 0 \\
  \comm{X_{\text{a},1}}{X_{\text{a},4}} &= - k_G X_{\text{a},4} \\
  \comm{X_{\text{a},1}}{X_{\text{a},6}} &= - k_G X_{\text{a},6} \\
  \comm{X_{\text{a},2}}{X_{\text{a},3}} &= - X_{\text{a},3} \\
  \begin{split} \label{eq:gompertz-autonomous-new-generator-1}
    \comm{X_{\text{a},2}}{X_{\text{a},4}} &= -\frac{1}{k_G} X_{\text{a},4} - t e^{-k_G t} \partial_t + \\
    &\quad + e^{-k_G t} W \left(k_G t \ln(W) + \ln(A) \ln(\ln(\frac{W}{A}))\right) \partial_W
  \end{split}\\
  \comm{X_{\text{a},2}}{X_{\text{a},6}} &= -(1 + k_G t) X_{\text{a},6} - e^{-k_G t} \ln(\ln(\frac{W}{A})) W \partial_W \label{eq:gompertz-autonomous-new-generator-2} \\
  \comm{X_{\text{a},3}}{X_{\text{a},4}} &= k_G \ln(A) X_{\text{a},6} \\
  \comm{X_{\text{a},3}}{X_{\text{a},6}} &= - X_{\text{a},6}\\
  \comm{X_{\text{a},4}}{X_{\text{a},6}} &= 0
\end{align}
the basis does not seem to be a full Lie algebra.
Assigning the additional generator
\begin{equation}
  X_{\text{a},7} = - \left(\comm{X_{\text{a},2}}{X_{\text{a},6}} + X_{\text{a},6}\right) = e^{-k_G t} W \left(k_G t + \ln(\ln(\frac{W}{A}))\right) \partial_W,
\end{equation}
\cref{eq:gompertz-autonomous-new-generator-1,eq:gompertz-autonomous-new-generator-2} can be written as
\begin{align}
  \comm{X_{\text{a},2}}{X_{\text{a},4}} &= -X_{\text{a},6} - X_{\text{a},7} \\
  \comm{X_{\text{a},2}}{X_{\text{a},6}} &= -k_G t e^{-k_G t} X_{\text{a},0} - X_{\text{a},4} + k_G \ln(A) X_{\text{a},7}
\end{align}
where \(X_{\text{a},0}\) is the trivial generator
\begin{equation}
  X_{\text{a},0} = \partial_t + \omega_a \partial_W = \partial_t - k_G \ln(\frac{W}{A}) W \partial_W.
\end{equation}
Looking at the additional Lie brackets
\begin{align}
  \comm{X_{\text{a},1}}{X_{\text{a},7}} &= k_G X_{\text{a},6} - k_G X_{\text{a},7} \\
  \comm{X_{\text{a},2}}{X_{\text{a},7}} &= - e^{-k_G t} W \left((k_G)^2 t^2 + 2 k_G t \ln(\ln(\frac{W}{A})) + \left(\ln(\ln(\frac{W}{A}))\right)^2 \right) \partial_W \\
  \comm{X_{\text{a},3}}{X_{\text{a},7}} &= X_{\text{a},6} - X_{\text{a},7} \\
  \comm{X_{\text{a},4}}{X_{\text{a},7}} &= 0 \\
  \comm{X_{\text{a},6}}{X_{\text{a},7}} &= \frac{e^{-2 k_G t} W}{\ln(\frac{W}{A})} \partial_W.
\end{align}
Thus it seems like two new generators are found.
This has to be explored further, but at this point some general calculations and computer implementations are probably the first step.
Further assignment of the additional generators



\subsection{The system Gompertz model}

For the system Gompertz model there is also only one parametrization of interest.
The linearized symmetry condition \labelcref{eq:linearized-first-order-symmetry} is for the system Gompertz model
\begin{subequations}\label{eq:gompertz-system-lin-symmetry-cond}
  \begin{align}
    \begin{split}\label{eq:gompertz-system-lin-symmetry-cond-a}
      \eta^1_t + W G \left(\eta^1_W - \xi_t\right) -k_G G \eta^1_G - W^2 G^2 \xi_W +&\\+ k_G W G^2 \xi_G - G \eta^1 - W \eta^2 &= 0 
    \end{split}\\
    \begin{split}\label{eq:gompertz-system-lin-symmetry-cond-b}
      \eta^2_t - k_G G \left(\eta^2_G - \xi_t\right) + W G \eta^2_G + k_G W G^2 \xi_W -&\\- (k_G)^2 G^2 \xi_G + k_G \eta^2 &= 0. 
    \end{split}
  \end{align}
\end{subequations}

\subsubsection{\texorpdfstring{\(k_G\)-independent symmetries}{Growth rate-independent symmetries}}

Decomposition of \cref{eq:gompertz-system-lin-symmetry-cond} in \(k_G\) gives the parameter independence determining equations
\begin{subequations}
  \begin{flalign}
    \labelcref{eq:gompertz-system-lin-symmetry-cond-a},\ & 1 : & \eta^1_t + W G \left(\eta^1_W - \xi_t\right) - W^2 G^2 \xi_W - G \eta^1 - W \eta^2 &= 0 && \FlLabel{eq:gompertz-system-det-kg-a}\\
    \labelcref{eq:gompertz-system-lin-symmetry-cond-a},\ & k_G : & - G \eta^1_G + W G^2 \xi_G &= 0 && \FlLabel{eq:gompertz-system-det-kg-b}\\
    \labelcref{eq:gompertz-system-lin-symmetry-cond-b},\ & 1 : & \eta^2_t + W G \eta^2_G &= 0. && \FlLabel{eq:gompertz-system-det-kg-c} \\
    \labelcref{eq:gompertz-system-lin-symmetry-cond-b},\ & k_G : & - G \left(\eta^2_G - \xi_t\right) + W G^2 \xi_W + \eta^2 &= 0. && \FlLabel{eq:gompertz-system-det-kg-d} \\
    \labelcref{eq:gompertz-system-lin-symmetry-cond-b},\ & (k_G)^2 : & - G^2 \xi_G &= 0. && \FlLabel{eq:gompertz-system-det-kg-e}
  \end{flalign}
\end{subequations}
From \cref{eq:gompertz-system-det-kg-e} it is clear that
\begin{equation}
  \xi = \xi(t, W),
\end{equation}
and thus from \cref{eq:gompertz-system-det-kg-b}
\begin{equation}
  \eta^1 = \eta^1(t, W).
\end{equation}
From \cref{eq:gompertz-system-det-kg-a}
\begin{equation}
  \eta^2 = \frac{1}{W}\eta^1_t + G \left(\eta^1_W - \xi_t - \frac{1}{W} \eta^1 \right) - W G^2 \xi_W,
\end{equation}
and thus \cref{eq:gompertz-system-det-kg-d} can be written as
\begin{equation}\label{eq:gompertz-system-det-kg-d-simple}
  \frac{1}{W}\eta^1_t + G \xi_t + 2 G^2 W \xi_W = 0.
\end{equation}
Since both \(\xi\) and \(\eta^1\) are not functions in \(G\), \cref{eq:gompertz-system-det-kg-d-simple} can be decomposed in \(G\) giving
\begin{align}
  \xi_t &= 0\\
  \xi_W &= 0\\
  \eta^1_t &= 0.
\end{align}
Thus
\begin{align}
  \xi &= c_1\\
  \eta^1 &= \eta^1(W)
  \eta^2 &= G \left(\eta^1_W(W) - \frac{1}{W} \eta^1(W) \right),
\end{align}
where \(c_1\) is an arbitrary constant, and \cref{eq:gompertz-system-det-kg-c} reduces to
\begin{equation}
  W G^2 \left(\eta^1_W - \frac{1}{W} \eta^1 \right)_W = 0.
\end{equation}
Thus the ODE
\begin{equation}
  \eta^1_W - \frac{1}{W} \eta^1 = c_2
\end{equation}
must hold for an arbitrary constant \(c_2\), which has the general solution
\begin{equation}
  \eta^1 = c_2 \ln(W) W + c_3 W.
\end{equation}
Thus any \(k_G\)-independent symmetry generator of the system Gompertz model \labelcref{eq:gompertz-system}\ghostref{eq:gompertz-system-a}\ghostref{eq:gompertz-system-b} must have the form
\begin{align}
  \xi &= c_1 \\
  \eta^1 &= c_2 \ln(W) W + c_3 W\\
  \eta^2 &= c_2 G
\end{align}
which is spanned by the generator basis
\begin{align}
  X_{\text{s},1} &= \partial_t \\
  X_{\text{s},2} &= \ln(W) W \partial_W + G \partial_G \\
  X_{\text{s},3} &= W \partial_W.
\end{align}
