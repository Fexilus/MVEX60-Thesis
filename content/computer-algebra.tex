\chapter{Computer algebra system} \label{app:computer-algebra}

% Fix description/listings interaction
% This breaks other parts of the document if defined in definitions.sty,
% so it is defined here instead.
% Based on: https://tex.stackexchange.com/a/93950
\makeatletter
  \let\orig@item\item

  \def\item{%
      \@ifnextchar{[}%
          {\lstinline@item}%
          {\orig@item}%
  }

  \begingroup
  \catcode`\]=\active
  \gdef\lstinline@item[{%
      \setbox0\hbox\bgroup
          \catcode`\]=\active
          \let]\lstinline@item@end
  }
  \endgroup

  \def\lstinline@item@end{%
      \egroup
      \orig@item[\usebox0]%
  }
\makeatother

\section*{symmetries}

Calculate symmetries of ODE:s symbolically.

\subsection*{class \lstinline{symmetries.Generator(xis, etas, total_space)}}

   A local coordinate representation of an infinitesimal generator.

   \subsubsection*{\lstinline{__call__(expr, jet_space=None)}}

      Apply the generator on an expression on a jet space.

\subsection*{class \lstinline{symmetries.JetSpace(base_coord, fibre_coord, degree)}}

   A local coordinate representation of a jet space.

   \subsubsection*{\lstinline{base_index(base_symbol)}}

      Returns the derivative index for a coordinate in the base space.

   \subsubsection*{\lstinline{extension(new_degree)}}

      Creates a jet space on the same total space of a higher degree.

   \subsubsection*{\lstinline{get_dependents()}}

      Return the dependent coordinates of the jet space.

   \subsubsection*{property \lstinline{original_total_space}}

      Return the coordinates of the total space on which the jet
      space is built

      Returns:
         A 2-tuple of lists of the coordinates of the base space and
         fiber respectively.

\subsection*{\lstinline{symmetries.decompose_generator(generator, basis)}}

   Decompose a generator by a basis of arbitrary constants or
   functions.

   Only generators linear in the basis is implemented.

\subsection*{\lstinline{symmetries.generator_on(total_space)}}

   Returns a initialization method for generators on the total space.

   Is meant to be used in code where several generators on the same
   space are used to reduce visual clutter.

\subsection*{\lstinline{symmetries.get_lin_symmetry_cond(diff_eqs, generator, jet_space,}\newline\lstinline{derivative_hints=None)}}

   Test if the linearized symmetry conditions hold differential
   equations.

   \begin{description}
      \item[Args:] \leavevmode
        \begin{description}
          \item[\lstinline{diff_eqs}:] A single or iterable of differential equations
            formulated in a jet space.

          \item[\lstinline{generator}:] An infinitesimal generator that takes an expression
            and a jet space and applies the prolongation on that jet
            space to the expression.

          \item[\lstinline{jet_space}:] The jet space on which the differential equations
            exist.

          \item[\lstinline{derivative_hints}:] If given, contains a single item or list
            (depending on the form of the differential equations) of the
            highest order derivatives to be solved for.
        \end{description}

      \item[Returns:]
        A single or list (depending on the form of the differential
        equations) of differential equations that must hold for the
        infinitesimal generator to generate a group of symmetries that
        constitute a symmetry.
   \end{description}

\subsection*{\lstinline{symmetries.get_prolongations(xis, etas, jet_space)}}

   Calculate the coefficients of a vector field prolonged over a jet
   space.

   The vector field is characterized by the coefficients of
   derivatives in the base space (xis) and the coefficients of
   derivatives in the fiber of the original fiber bundle (etas) from
   which the jet space is created.

\subsection*{\lstinline{symmetries.lie_bracket(generator1, generator2)}}

   The Lie bracket of two generators in the same coordinate system.

\subsection*{\lstinline{symmetries.total_derivative(jet_exp, coordinate, domain)}}

   The total derivative of an expression in a coordinate.


\section*{\lstinline{symmetries.ansatz}}

Make ansätze for generators.

\subsection*{\lstinline{symmetries.ansatz.create_poly_ansatz(jet_space, degree=1)}}

   Create an infinitesimal generator that is polynomial in the
   components of a given jet space.
