\chapter{Computer algebra system} \label{app:computer-algebra}

% Fix description/listings interaction
% This breaks other parts of the document if defined in definitions.sty,
% so it is defined here instead.
% Based on: https://tex.stackexchange.com/a/93950
\makeatletter
  \let\orig@item\item

  \def\item{%
      \@ifnextchar{[}%
          {\lstinline@item}%
          {\orig@item}%
  }

  \begingroup
  \catcode`\]=\active
  \gdef\lstinline@item[{%
      \setbox0\hbox\bgroup
          \catcode`\]=\active
          \let]\lstinline@item@end
  }
  \endgroup

  \def\lstinline@item@end{%
      \egroup
      \orig@item[\usebox0]%
  }
\makeatother

% Raise asterisks in the code listings
% Based on: https://tex.stackexchange.com/a/303477
\makeatletter
\lst@CCPutMacro
    \lst@ProcessOther {"2A}{%
      \lst@ttfamily 
         {\raisebox{2pt}{*}}% used with ttfamily
         \textasteriskcentered}% used with other fonts
    \@empty\z@\@empty
\makeatother  

% Listings settings
\lstset{
   alsoletter={.},
   emph={
      sympy.Expr,
      ValueError,
      Generator,
      __call__,
      get_jet_space_basis,
      get_tangent_field,
      JetSpace,
      base_index,
      dependents,
      extension,
      original_total_space,
      decompose_generator,
      generator_on,
      get_lin_symmetry_cond,
      get_prolongations,
      lie_bracket,
      total_derivative,
      create_poly_ansatz,
      derivatives_sort_key,
      iter_wrapper,
      optional_iter,
      replace_consts,
      zip_strict,
   },
   emphstyle=\textbf,
   basicstyle=\linespread{0.83}\ttfamily,
}

\section*{\lstinline{symmetries}}

Calculate symmetries of ODE:s symbolically.

\subsection*{class \lstinline{symmetries.Generator(xis, etas, total_space)}}

   A local coordinate representation of an infinitesimal generator.

   \begin{description}
      \item[Parameters:] \leavevmode
        \begin{description}
            \item[\lstinline{xis (list[sympy.Expr])} -] The base space components of the tangent field.
            \item[\lstinline{etas (list[sympy.Expr])} -] The fiber components of the tangent field.
            \item[\lstinline{total_space (tuple[list[sympy.Expr], list["sympy.Expr"]])} -] The base\newline vectors of the total space on which the generator acts.
        \end{description}
   \end{description}

   \subsubsection*{\lstinline{__call__(expr, jet_space=None)}}

      Apply the generator on an expression on a jet space.

      \begin{description}
         \item[Parameters:] \leavevmode
           \begin{description}
               \item[\lstinline{expr (sympy.Expr)} -] The expression to apply the generator on.
               \item[\lstinline{jet_space (JetSpace, optional)} -] The jet space in which the expression lives.
           \end{description}
         \item[Returns:] The expression after application of the generator.
         \item[Return type:] \lstinline{sympy.Expr}
      \end{description}
   
   \subsubsection*{\lstinline{get_jet_space_basis(degree=0)}}

      Return the basis of a jet space on which the generator can act.

      The ordering is the same as the ordering of \lstinline{get_tangent_field()}.

      \begin{description}
         \item[Parameters:] \leavevmode
           \begin{description}
               \item[\lstinline{degree (int, optional)} -] The degree of the jet space.
           \end{description}
         \item[Returns:] The expressions corresponding to the basis vectors of the jet space.
         \item[Return type:] \lstinline{list[sympy.Expr]}
      \end{description}

   \subsubsection*{\lstinline{get_tangent_field(degree=0)}}

      Return the corresponding prolonged tangent field of the
      generator.

      The ordering is the same as the ordering of \lstinline{get_jet_space_basis()}.

      \begin{description}
         \item[Parameters:] \leavevmode
           \begin{description}
               \item[\lstinline{degree (int, optional)} -] The degree of the prolongation.
           \end{description}
         \item[Returns:] The expressions corresponding to the components of the (possibly prolonged) tangent field.
         \item[Return type:] \lstinline{list[sympy.Expr]}
      \end{description}

\subsection*{class \lstinline{symmetries.JetSpace(base_coord, fiber_coord, degree)}}

   A local coordinate representation of a jet space.

   \begin{description}
      \item[Parameters:] \leavevmode
        \begin{description}
            \item[\lstinline{base_coord (list[sympy.Expr])} -] The basis vectors of the base space.
            \item[\lstinline{fiber_coord (list[sympy.Expr])} -] The basis vectors of the fibers.
            \item[\lstinline{degree (int)} -] The degree of the created jet space. A degree of 0 corresponds to the total space.
        \end{description}     
   \end{description}

   \subsubsection*{\lstinline{base_index(base_symbol)}}

      Returns the derivative index for a coordinate in the base space.

      \begin{description}
         \item[Parameters:] \leavevmode
           \begin{description}
               \item[\lstinline{base_symbol (sympy.Expr)} -] The symbol to find the base index of.
           \end{description}
         \item[Returns:] The multiindex of the desired symbol in the base space.
         \item[Return type:] \lstinline{tuple[int, ...]} 
      \end{description}

   \subsubsection*{property \lstinline{dependents}}

      The fibers of the total space on which the jet space is built.

      \begin{description}
         \item[Returns:] The symbols of the original fibers.
         \item[Return type:] \lstinline{list[sympy.Expr]}
      \end{description}
   
   \subsubsection*{\lstinline{extension(new_degree)}}

      Creates a jet space on the same total space of a higher degree.

      \begin{description}
         \item[Parameters:] \leavevmode
           \begin{description}
               \item[\lstinline{new_degree (int)} -] The degree of the extended jet space. Must be higher than the current degree.
           \end{description}
         \item[Returns:] A deep copy of the jet space, with additional jet fibers of higher degree.
         \item[Return type:] \lstinline{JetSpace}
      \end{description}

   \subsubsection*{property \lstinline{original_total_space}}

      Return the coordinates of the total space on which the jet
      space is built

      \begin{description}
         \item[Returns:] The coordinates of the base space and fiber.
         \item[Return type:] \lstinline{tuple[list[sympy.Expr], list[sympy.Expr]]}
      \end{description}         

\subsection*{\lstinline{symmetries.decompose_generator(generator, basis)}}

   Decompose a generator by a basis of arbitrary constants or
   functions.

   Only decomposition of generators linear in the basis is
   implemented.

   \begin{description}
      \item[Parameters:] \leavevmode
        \begin{description}
            \item[\lstinline{generator (Generator)} -] The generator to decompose.
            \item[\lstinline{basis (list[sympy.Expr])} -] The arbitrary constants or functions in which the generator can be decomposed.
        \end{description}
      \item[Returns:] The generators that span the space the input generator was in.
      \item[Return type:] \lstinline{list[Generator]}
   \end{description}

\subsection*{\lstinline{symmetries.generator_on(total_space)}}

   Returns a initialization method for generators on the total space.

   Is meant to be used to reduce visual clutter in code where several
   generators on the same space are used.

   \begin{description}
      \item[Parameters:] \leavevmode
        \begin{description}
            \item[\lstinline{total_space (tuple[list[sympy.Expr], list[sympy.Expr]])} -] The base\newline vectors of the total space on which the generator acts.
        \end{description}
      \item[Returns:] A generator subclass without the total space argument in the initializer.
      \item[Return type:] \lstinline{Generator}
   \end{description}

\subsection*{\lstinline{symmetries.get_lin_symmetry_cond(diff_eqs, generator, jet_space,}\newline\lstinline{derivative_hints=None)}}

   Test if the linearized symmetry conditions hold differential
   equations.

   \begin{description}
      \item[Parameters:] \leavevmode
        \begin{description}
            \item[\lstinline{diff_eqs (sympy.Expr or list[sympy.Expr])} -] The differential equation(s)\newline expressed in jet space notation.
            \item[\lstinline{generator (Generator)} -] The generator corresponding to the Lie group of transformations to be tested.
            \item[\lstinline{jet_space (JetSpace)} -] The jet space on which the differential equations exist.
            \item[\lstinline{derivative_hints (sympy.Expr or list[sympy.Expr])} -] The highest order\newline derivative(s) that the differential equation(s) can be solved for.
        \end{description}

      \item[Returns:] The differential equation(s) that must hold for the infinitesimal generator to generate a Lie group of symmetries. The differential equations are expressed in jet space notation.
      \item[Return type:] \lstinline{sympy.Expr or list[sympy.Expr]}
   \end{description}

\subsection*{\lstinline{symmetries.get_prolongations(xis, etas, jet_space)}}

   Calculate the coefficients of a vector field prolonged over a jet
   space.

   The vector field is characterized by the coefficients of
   derivatives in the base space (xis) and the coefficients of
   derivatives in the fiber of the original fiber bundle (etas) from
   which the jet space is created.

   \begin{description}
      \item[Parameters:] \leavevmode
        \begin{description}
            \item[\lstinline{xis (list[sympy.Expr])} -] The base space components of the tangent field.
            \item[\lstinline{etas (list[sympy.Expr])} -] The fiber components of the tangent field.
            \item[\lstinline{jet_space (JetSpace)} -] The jet space on which the prolonged tangent field will be calculated.
        \end{description}
      \item[Returns:] The prolonged fiber expressions, ordered firstly by original fiber and secondly by corresponding derivative multiindex.
      \item[Return type:] \lstinline{dict[str, dict[tuple[int, ...], sympy.Expr]]}
   \end{description}

\subsection*{\lstinline{symmetries.lie_bracket(generator1, generator2)}}

   The Lie bracket of two generators in the same coordinate system.

   \begin{description}
      \item[Parameters:] \leavevmode
        \begin{description}
            \item[\lstinline{generator1 (Generator)} -] The left generator in the Lie bracket.
            \item[\lstinline{generator2 (Generator)} -] The right generator in the Lie bracket.
        \end{description}
      \item[Returns:] The Lie bracket of the two generators.
      \item[Return type:] \lstinline{Generator}
   \end{description}

\subsection*{\lstinline{symmetries.total_derivative(jet_exp, coordinate, domain)}}

   The total derivative of an expression in a coordinate.

   \begin{description}
      \item[Parameters:] \leavevmode
        \begin{description}
            \item[\lstinline{jet_exp (sympy.Expr)} -] The expression to be derived.
            \item[\lstinline{coordinate (sympy.Expr)} -] The coordinate to derive in. Should be one of the symbols of the base space.
            \item[\lstinline{domain (JetSpace)} -] The jet space in which \lstinline{jet_exp} exists.
        \end{description}
      \item[Returns:] The derived expression. This expression exists in the the (once) extended jet space compared to the domain of the derivative.
      \item[Return type:] \lstinline{sympy.Expr}
   \end{description}

\section*{\lstinline{symmetries.ansatz}}

Make ansätze for generators.

\subsection*{\lstinline{symmetries.ansatz.create_poly_ansatz(jet_space, degree=1)}}

   Create an infinitesimal generator that is polynomial in the
   components of a given jet space.

   \begin{description}
      \item[Parameters:] \leavevmode
        \begin{description}
            \item[\lstinline{jet_space (JetSpace)} -] The jet space on which the generator can act.
            \item[\lstinline{degree (int, optional)} -] The degree of the polynomials in all generator components.
        \end{description}
      \item[Returns:] The generator and arbitrary constants defined by the ansatz.
      \item[Return type:] \lstinline{tuple[Generator, list[sympy.Expr]]}
   \end{description}

\section*{\lstinline{symmetries.utils}}

Utilities for the main symmetries package.

\subsection*{\lstinline{symmetries.utils.derivatives_sort_key(function_order=None,}\newline\lstinline{argument_order=None)}}

   Ad hoc sort key for derivatives.

   \begin{description}
      \item[Parameters:] \leavevmode
        \begin{description}
            \item[\lstinline{function_order (list[sympy.Derivative], optional)} -] An ordering of the\newline functions.
            \item[\lstinline{argument_order (list[sympy.Expr], optional)} -] An ordering of the\newline arguments.
        \end{description}
      \item[Returns:] A sorting key for \lstinline{sympy.Derivative} that sorts firstly by function, secondly by derivative degree and thirdly by derivative arguments.
   \end{description}

\subsection*{\lstinline{symmetries.utils.iter_wrapper(possible_iter)}}

   Ensures that the argument can be treated as an iterable.

   \begin{description}
      \item[Parameters:] \leavevmode
        \begin{description}
            \item[\lstinline{possible_iter} -] A possible iterable.
        \end{description}
      \item[Returns:] A guaranteed iterable.
   \end{description}      

\subsection*{\lstinline{symmetries.utils.optional_iter(func)}}

   Decorator that allows the first argument and return value to be
   optionally treated as iterables.

   \begin{description}
      \item[Parameters:] \leavevmode
        \begin{description}
            \item[\lstinline{func} -] The function should take an iterable as first argument and return an iterable.
        \end{description}
      \item[Returns:] The decorated function will accept either a single element or an iterable as first argument, and will return a single element or an iterable depending on the input type.
   \end{description}

\subsection*{\lstinline{symmetries.utils.replace_consts(exprs, new_const_name)}}

   Replace arbitrary constant from eg. \lstinline{sympy.dsolve}.

   \begin{description}
      \item[Parameters:] \leavevmode
        \begin{description}
            \item[\lstinline{exprs (sympy.Expr or list[sympy.Expr])} -] The expression(s) to\newline replace constants in.
            \item[\lstinline{new_const_name (str)} -] The new base name to give to the constants. E.g.\newline \lstinline{new_const_name = r"\\alpha"} will result in constants \lstinline|r"\alpha_{1}"|,\newline \lstinline|r"\alpha_{2}"| etc.
        \end{description}
      \item[Returns:] The new expressions and all the new constants.
      \item[Return type:] \lstinline{tuple[sympy.Expr or list[sympy.Expr], list[sympy.Expr]]}
   \end{description}

\subsection*{\lstinline{symmetries.utils.zip_strict(*iters)}}

   Zip two iterables and ensure that they are equal. Will be replaced
   with \lstinline{strict=True} argument in Python 3.10.

   \begin{description}
      \item[Parameters:] \leavevmode
        \begin{description}
          \item[\lstinline{*iters} -] Any number of iterables.
        \end{description}
      \item[Returns:] An iterable through tuples of the ingoing iterables.
      \item[Raises:] \leavevmode
      \begin{description}
        \item[\lstinline{ValueError} -] if the iterables have different lengths.
      \end{description}
   \end{description}     
